\documentclass[journal,12pt,twocolumn]{IEEEtran}
\usepackage{setspace}
\usepackage{gensymb}
\singlespacing
\usepackage[cmex10]{amsmath}
\usepackage{amsthm}
\usepackage{mathrsfs}
\usepackage{txfonts}
\usepackage{stfloats}
\usepackage{bm}
\usepackage[table,xcdraw]{xcolor}
\usepackage{cite}
\usepackage{cases}
\usepackage{subfig}
\usepackage{longtable}
\usepackage{multirow}
%\usepackage{algorithm}
\usepackage{enumitem}
\usepackage{mathtools}
\usepackage{steinmetz}
\usepackage{tikz}
\usepackage{circuitikz}
\usepackage{verbatim}
\usepackage{tfrupee}
\usepackage[breaklinks=true]{hyperref}
%\usepackage{stmaryrd}
\usepackage{tkz-euclide} % loads  TikZ and tkz-base
%\usetkzobj{all}
\usetikzlibrary{calc,math}
\usepackage{listings}
\usepackage{color}                                            %%
\usepackage{array}                                            %%
\usepackage{longtable}                                        %%
\usepackage{calc}                                             %%
\usepackage{multirow}                                         %%
\usepackage{hhline}                                           %%
\usepackage{ifthen}                                           %%
%optionally (for landscape tables embedded in another document): %%
\usepackage[]{caption}
\usepackage{lscape}     
\usepackage{multicol}
\usepackage{chngcntr}
\usepackage{enumerate}
%\usepackage{placeins}
%\usepackage{float}
%\restylefloat{table}
%\usepackage{wasysym}
%\newcounter{MYtempeqncnt}
\DeclareMathOperator*{\Res}{Res}
%\renewcommand{\baselinestretch}{2}
\renewcommand\thesection{\arabic{section}}
\renewcommand\thesubsection{\thesection.\arabic{subsection}}
\renewcommand\thesubsubsection{\thesubsection.\arabic{subsubsection}}
\newcommand\numberthis{\addtocounter{equation}{1}\tag{\theequation}}
\renewcommand\thesectiondis{\arabic{section}}
\renewcommand\thesubsectiondis{\thesectiondis.\arabic{subsection}}
\renewcommand\thesubsubsectiondis{\thesubsectiondis.\arabic{subsubsection}}

% correct bad hyphenation here
\hyphenation{op-tical net-works semi-conduc-tor}
\def\inputGnumericTable{}                                 %%

\lstset{
	%language=C,
	frame=single, 
	breaklines=true,
	columns=fullflexible
}


\begin{document}
	%
	
	
	\newtheorem{theorem}{Theorem}[section]
	\newtheorem{problem}{Problem}
	\newtheorem{proposition}{Proposition}[section]
	\newtheorem{lemma}{Lemma}[section]
	\newtheorem{corollary}[theorem]{Corollary}
	\newtheorem{example}{Example}[section]
	\newtheorem{definition}[problem]{Definition}
	
	\newcommand{\BEQA}{\begin{eqnarray}}
	\newcommand{\EEQA}{\end{eqnarray}}
	\newcommand{\define}{\stackrel{\triangle}{=}}
	\bibliographystyle{IEEEtran}
	%\bibliographystyle{ieeetr}
	\providecommand{\mbf}{\mathbf}
	\providecommand{\pr}[1]{\ensuremath{\Pr\left(#1\right)}}
	\providecommand{\qfunc}[1]{\ensuremath{Q\left(#1\right)}}
	\providecommand{\sbrak}[1]{\ensuremath{{}\left[#1\right]}}
	\providecommand{\lsbrak}[1]{\ensuremath{{}\left[#1\right.}}
	\providecommand{\rsbrak}[1]{\ensuremath{{}\left.#1\right]}}
	\providecommand{\brak}[1]{\ensuremath{\left(#1\right)}}
	\providecommand{\lbrak}[1]{\ensuremath{\left(#1\right.}}
	\providecommand{\rbrak}[1]{\ensuremath{\left.#1\right)}}
	\providecommand{\cbrak}[1]{\ensuremath{\left\{#1\right\}}}
	\providecommand{\lcbrak}[1]{\ensuremath{\left\{#1\right.}}
	\providecommand{\rcbrak}[1]{\ensuremath{\left.#1\right\}}}
	\theoremstyle{remark}
	\newtheorem{rem}{Remark}
	\newcommand{\sgn}{\mathop{\mathrm{sgn}}}
	\providecommand{\abs}[1]{\left\vert#1\right\vert}
	\providecommand{\res}[1]{\Res\displaylimits_{#1}} 
	\providecommand{\norm}[1]{\left\lVert#1\right\rVert}
	%\providecommand{\norm}[1]{\lVert#1\rVert}
	\providecommand{\mtx}[1]{\mathbf{#1}}
	\providecommand{\mean}[1]{E\left[ #1 \right]}
	\providecommand{\fourier}{\overset{\mathcal{F}}{ \rightleftharpoons}}
	%\providecommand{\hilbert}{\overset{\mathcal{H}}{ \rightleftharpoons}}
	\providecommand{\system}{\overset{\mathcal{H}}{ \longleftrightarrow}}
	%\newcommand{\solution}[2]{\textbf{Solution:}{#1}}
	\newcommand{\solution}{\noindent \textbf{Solution: }}
	\newcommand{\cosec}{\,\text{cosec}\,}
	\providecommand{\dec}[2]{\ensuremath{\overset{#1}{\underset{#2}{\gtrless}}}}
	\newcommand{\myvec}[1]{\ensuremath{\begin{pmatrix}#1\end{pmatrix}}}
	\newcommand{\mydet}[1]{\ensuremath{\begin{vmatrix}#1\end{vmatrix}}}
	\numberwithin{equation}{subsection}
	\makeatletter
	\@addtoreset{figure}{problem}
	\makeatother
	\let\StandardTheFigure\thefigure
	\let\vec\mathbf
	\renewcommand{\thefigure}{\theproblem}
	\def\putbox#1#2#3{\makebox[0in][l]{\makebox[#1][l]{}\raisebox{\baselineskip}[0in][0in]{\raisebox{#2}[0in][0in]{#3}}}}
	\def\rightbox#1{\makebox[0in][r]{#1}}
	\def\centbox#1{\makebox[0in]{#1}}
	\def\topbox#1{\raisebox{-\baselineskip}[0in][0in]{#1}}
	\def\midbox#1{\raisebox{-0.5\baselineskip}[0in][0in]{#1}}
	\vspace{3cm}
	\title{Matrix Theory Assignment 12}
	\author{Ritesh Kumar \\ EE20RESCH11005}
	
	
	\maketitle
	\newpage
	%\tableofcontents
	\bigskip
	\renewcommand{\thefigure}{\theenumi}
	\renewcommand{\thetable}{\theenumi}
	\counterwithout{figure}{section}
	\counterwithout{figure}{subsection}
	\date{Today}
	

\begin{abstract}
This problem is all about to to introducing the concept of linear algebra over a filed.
\end{abstract}
All the codes for this document can be found at
\begin{lstlisting}
https://github.com/Ritesh622/Assignment_EE5609/tree/master/Assignment_12
\end{lstlisting}
\section{Problem}
If $a$ and $b$ are element of a filed $\mathbb{F}$ and $ a \neq 0$, show that the ploynomial $ 1, ax+b, \brak{ax+b}^2, \brak{ax+b}^3,  \dots $ form a basis of $\mathbb{F}[x].$
\section{solution}
 Let consider we have a set $S$ such that,
 \begin{align}
 S = \left \{    1, ax+b, \brak{ax+b}^2, \brak{ax+b}^3,  \dots  \right \}
 \end{align}
 And let $\left \langle S \right \rangle$ be the subspace, that is spanned by $S$.
 \begin{align}
 \intertext{Since}
  1 \in S\\
 \intertext{and}
  ax+b \in S,\\
  \implies  b.1 + \frac{1}{a}\brak{b +ax} \in \left \langle S \right \rangle \\
  \intertext{and hence, it follows}
   \implies x \in \left \langle S \right \rangle 
 \end{align}
 
 Now to prove
 \begin{align}
  x^2 \in \left \langle S \right \rangle \\
 \intertext{let consider another element form $S$ which is}
    \brak{ax+b}^2
\end{align}
Subtracting $1.a^2 +2.a.b.x$ from $\brak{ax+b}^2$
 \begin{align}
 \implies \brak{ax+b}^2 - a^2 - 2.a.b.x = a^2.x^2 \label{2.2}\\
 \implies a^2.x^2 \in \left \langle S \right \rangle\\
 \implies   \frac{1}{a^2} . a^2. x^2 \in S.\\
 \implies x^2 \in  \left \langle S \right \rangle .
 \end{align}
 Now, Thus  Hence using this concept with higher degree we can prove that,
 \begin{align}
   x^n \in  \left \langle S \right \rangle, \forall  n
  \end{align}
Consider,
 \begin{align}
 S' =   \left \{    1, x, x^2, x^3,  \dots  \right \} \label{2.3}
   \end{align}
 Hence we can say that, \eqref{2.3} span the space of all polynomials which form with the help of
 \begin{align} 
  \brak{ax + b}^n
  \end{align}
   
Hence we conclude   that $S$ spans the space of all polynomials.
We can summarize our procedure step by step using table\ref{t1}.

\renewcommand{\thetable}{1}
%\begin{longtable*} 
%	\tiny
%	{
%	\caption{Step for the solution}
%			\label{t1}
%	\begin{tabular}{|p{1cm}|p{4cm}|p{3cm}|}	
%			\hline
%				\centering
%			\textbf{Sr. No.} & \textbf{Description} & \textbf{ Mathematical  representation} \\ 
%			\hline
%			\centering 
%			1. & Consider  a set $S$ & $S = \left \{    1, ax+b, \dots  \right \}$   \\
%			\hline
%			\centering
%			2. & Provide a proof that subset $S$ span the subspace $\left \langle S \right \rangle$  & Given element are $\in S $  \\
%			\hline
%			\centering
%			3. & Repeat  step 2 for the higher degree of polynomial also lie in the subspace and the also lie in the subset $S$. & Given element are $\in S $  \\
%			\hline	
%			\centering
%			4. &After providing proof for all element $\in S$ find the basis  . & $ S'=\left \{    1, x, x^2, x^3,  \dots  \right \} $  \\
%			\hline
%			\centering
%			5. & show the element $\in S'$ are  able to form all element $S$ over $\mathbb{F}$ . & Hence $S$ form basis of $\mathbb{F}$    \\
%			\hline
%		\end{tabular} 
%	}
%\end{longtable*}
%











\begin{multicols}{3}
\begin{table*}[htb!]%\vspace{10pt}
	\centering	 
	\begin{scriptsize}
		\caption{Step for the solution}
		\label{t1}	 
		\begin{tabular}{|p{1cm}|p{4cm}|p{4cm}|}
			\hline
			\centering
			\textbf{\ \ Sr. No} &\textbf{Description} &  \textbf{ Mathematical  representation}  \\ 
			
			\hline
			\centering
			1. & Consider  a set $S$ & $S = \left \{    1, ax+b, \dots  \right \}$ \\
			\hline 
			\centering
			2. & Provide a proof that subset $S$ span the subspace $\left \langle S \right \rangle$  & Since
			$1 \in S$
			and
			$ax+b \in S,$
			$\implies  b.1 + \frac{1}{a}\brak{b +ax} \in \left \langle S \right \rangle$ 
			  $\implies x \in \left \langle S \right \rangle$  Given element are $\in S $  \\
			\hline
			\centering
			3. & Repeat  step 2 for the higher degree of polynomial also lie in the subspace and the also lie in the subset $S$. & Since $\brak{ax+b}^2 \in S $   $ \implies \brak{ax+b}^2 - a^2 - 2.a.b.x = a^2.x^2 
			\implies a^2.x^2 \in \left \langle S \right \rangle
			\implies   \frac{1}{a^2} . a^2. x^2 \in S.
			\implies x^2 \in  \left \langle S \right \rangle$
			 Given element are $\in S $ \\
			\hline
			\centering
			4. &After providing proof for all element $\in S$ find the basis  . & $ S'=\left \{    1, x, x^2, x^3,  \dots  \right \} $ \\
			\hline
			\centering
		5. & Show the element $\in S'$ are  able to form all element $S$ over $\mathbb{F}$ . & Hence $S$ form basis of $\mathbb{F}$ \\
			\hline	
		\end{tabular}
	\end{scriptsize}
\end{table*}
\end{multicols}









\end{document}