\documentclass[journal,12pt,twocolumn]{IEEEtran}
\usepackage{setspace}
\usepackage{gensymb}
\singlespacing
\usepackage[cmex10]{amsmath}
\usepackage{amsthm}
\usepackage{mathrsfs}
\usepackage{txfonts}
\usepackage{stfloats}
\usepackage{bm}
\usepackage[table,xcdraw]{xcolor}
\usepackage{cite}
\usepackage{cases}
\usepackage{subfig}
\usepackage{longtable}
\usepackage{multirow}
%\usepackage{algorithm}
\usepackage{enumitem}
\usepackage{mathtools}
\usepackage{steinmetz}
\usepackage{tikz}
\usepackage{circuitikz}
\usepackage{verbatim}
\usepackage{tfrupee}
\usepackage[breaklinks=true]{hyperref}
%\usepackage{stmaryrd}
\usepackage{tkz-euclide} % loads  TikZ and tkz-base
%\usetkzobj{all}
\usetikzlibrary{calc,math}
\usepackage{listings}
\usepackage{color}                                            %%
\usepackage{array}                                            %%
\usepackage{longtable}                                        %%
\usepackage{calc}                                             %%
\usepackage{multirow}                                         %%
\usepackage{hhline}                                           %%
\usepackage{ifthen}                                           %%
%optionally (for landscape tables embedded in another document): %%
\usepackage[]{caption}
\usepackage{lscape}     
\usepackage{multicol}
\usepackage{chngcntr}
\usepackage{enumerate}
%\usepackage{placeins}
%\usepackage{float}
%\restylefloat{table}
%\usepackage{wasysym}
%\newcounter{MYtempeqncnt}
\DeclareMathOperator*{\Res}{Res}
%\renewcommand{\baselinestretch}{2}
\renewcommand\thesection{\arabic{section}}
\renewcommand\thesubsection{\thesection.\arabic{subsection}}
\renewcommand\thesubsubsection{\thesubsection.\arabic{subsubsection}}
\newcommand\numberthis{\addtocounter{equation}{1}\tag{\theequation}}
\renewcommand\thesectiondis{\arabic{section}}
\renewcommand\thesubsectiondis{\thesectiondis.\arabic{subsection}}
\renewcommand\thesubsubsectiondis{\thesubsectiondis.\arabic{subsubsection}}
\newcommand\bigSigma[1][17]{\mbox{\fontsize{#1}{0}\selectfont$\Sigma$}}
% correct bad hyphenation here
\hyphenation{op-tical net-works semi-conduc-tor}
\def\inputGnumericTable{}                                 %%

\lstset{
	%language=C,
	frame=single, 
	breaklines=true,
	columns=fullflexible
}


\begin{document}
	%
	
	
	\newtheorem{theorem}{Theorem}[section]
	\newtheorem{problem}{Problem}
	\newtheorem{proposition}{Proposition}[section]
	\newtheorem{lemma}{Lemma}[section]
	\newtheorem{corollary}[theorem]{Corollary}
	\newtheorem{example}{Example}[section]
	\newtheorem{definition}[problem]{Definition}
	\newcommand\bigZero[1][17]{\mbox{\fontsize{#1}{0}\selectfont$0$}}
	\newcommand{\BEQA}{\begin{eqnarray}}
	\newcommand{\EEQA}{\end{eqnarray}}
	\newcommand{\define}{\stackrel{\triangle}{=}}
	\bibliographystyle{IEEEtran}
	%\bibliographystyle{ieeetr}
	\providecommand{\mbf}{\mathbf}
	\providecommand{\pr}[1]{\ensuremath{\Pr\left(#1\right)}}
	\providecommand{\qfunc}[1]{\ensuremath{Q\left(#1\right)}}
	\providecommand{\sbrak}[1]{\ensuremath{{}\left[#1\right]}}
	\providecommand{\lsbrak}[1]{\ensuremath{{}\left[#1\right.}}
	\providecommand{\rsbrak}[1]{\ensuremath{{}\left.#1\right]}}
	\providecommand{\brak}[1]{\ensuremath{\left(#1\right)}}
	\providecommand{\lbrak}[1]{\ensuremath{\left(#1\right.}}
	\providecommand{\rbrak}[1]{\ensuremath{\left.#1\right)}}
	\providecommand{\cbrak}[1]{\ensuremath{\left\{#1\right\}}}
	\providecommand{\lcbrak}[1]{\ensuremath{\left\{#1\right.}}
	\providecommand{\rcbrak}[1]{\ensuremath{\left.#1\right\}}}
	\theoremstyle{remark}
	\newtheorem{rem}{Remark}
	\newcommand{\sgn}{\mathop{\mathrm{sgn}}}
	\providecommand{\abs}[1]{\left\vert#1\right\vert}
	\providecommand{\res}[1]{\Res\displaylimits_{#1}} 
	\providecommand{\norm}[1]{\left\lVert#1\right\rVert}
	%\providecommand{\norm}[1]{\lVert#1\rVert}
	\providecommand{\mtx}[1]{\mathbf{#1}}
	\providecommand{\mean}[1]{E\left[ #1 \right]}
	\providecommand{\fourier}{\overset{\mathcal{F}}{ \rightleftharpoons}}
	%\providecommand{\hilbert}{\overset{\mathcal{H}}{ \rightleftharpoons}}
	\providecommand{\system}{\overset{\mathcal{H}}{ \longleftrightarrow}}
	%\newcommand{\solution}[2]{\textbf{Solution:}{#1}}
	\newcommand{\solution}{\noindent \textbf{Solution: }}
	\newcommand{\cosec}{\,\text{cosec}\,}
	\providecommand{\dec}[2]{\ensuremath{\overset{#1}{\underset{#2}{\gtrless}}}}
	\newcommand{\myvec}[1]{\ensuremath{\begin{pmatrix}#1\end{pmatrix}}}
	\newcommand{\mydet}[1]{\ensuremath{\begin{vmatrix}#1\end{vmatrix}}}
	\numberwithin{equation}{subsection}
	\makeatletter
	\@addtoreset{figure}{problem}
	\makeatother
	\let\StandardTheFigure\thefigure
	\let\vec\mathbf
	\renewcommand{\thefigure}{\theproblem}
	\def\putbox#1#2#3{\makebox[0in][l]{\makebox[#1][l]{}\raisebox{\baselineskip}[0in][0in]{\raisebox{#2}[0in][0in]{#3}}}}
	\def\rightbox#1{\makebox[0in][r]{#1}}
	\def\centbox#1{\makebox[0in]{#1}}
	\def\topbox#1{\raisebox{-\baselineskip}[0in][0in]{#1}}
	\def\midbox#1{\raisebox{-0.5\baselineskip}[0in][0in]{#1}}
	\vspace{3cm}
	\title{Matrix Theory Assignment 15}
	\author{Ritesh Kumar \\ EE20RESCH11005}
	
	
	\maketitle
	\newpage
	%\tableofcontents
	\bigskip
	\renewcommand{\thefigure}{\theenumi}
	\renewcommand{\thetable}{\theenumi}
	\counterwithout{figure}{section}
	\counterwithout{figure}{subsection}
	\date{Today}
	
	
	\begin{abstract}
		This problem is all about to  introducing the concept of characteristic polynomial over a filed.
	\end{abstract}
	All the codes for this document can be found at
	\begin{lstlisting}
	https://github.com/Ritesh622/Assignment_EE5609/tree/master/Assignment_15
	\end{lstlisting}
	\section{Problem}
	Let $\mathbf{V}$ be the space of $ n \times n$  matrices over a field $\mathbb{F}$, and let $A$ be a fixed $n \times n $ matrix over field $\mathbb{F}$. Define a linear operator $\mathbf{T}$ on $\mathbf{V}$  by the equation 
	\begin{align}
	\mathbf{T}(\vec{B}) = \vec{A}\vec{B} - \vec{B}\vec{A}
	\end{align}
	Prove that if $\vec{A}$  is a nilpotent matrix, then $\mathbf{T}$ is a nilpotent operator.
	
	\section{solution}
	Since $\vec{A}$ is a nilpotent  matrix, hence for a positive value $K$:
	\begin{align}
	\vec{A}^K = \vec{0}
	\end{align} 
	Now we have 
	\begin{align}
	\mathbf{T}(\vec{B}) = \vec{A}\vec{\vec{B}} - \vec{B}\vec{A}\\
	\implies  \mathbf{T}^2(\vec{B}) = \mathbf{T}(\mathbf{T}(\vec{B})) =\mathbf{T}(\vec{A}\vec{B} -\vec{B}\vec{A})\label{2.3}	
	\end{align}
	\begin{multline}
	\mathbf{T}(\vec{A}\vec{B} -\vec{B}\vec{A}) = \vec{A}^2\vec{B} - \vec{A}\vec{B}\vec{A} -\vec{A}\vec{B}\vec{A} + \vec{B}\vec{A}^2 \\= \vec{A}^2 -2\vec{A}\vec{B}\vec{A} +\vec{B}\vec{A}^2
	\end{multline}
	\begin{multline}
	\implies\mathbf{T}^3(\vec{B}) = \mathbf{T}(\mathbf{T}^2(\vec{B})) \\= \vec{A}^3 - 3\vec{A}^2\vec{B}\vec{A} + 3\vec{A}\vec{B}\vec{A}^2 - \vec{B}\vec{A}^3 \label{2.5}
	\end{multline}
	Hence, from \eqref{2.5} we can say that, as we are increasing the power of operator $\mathbf{T}$, the power of $\vec{A}$ is also increasing in every terms. Hence there exist   a value $P$  such that :
	\begin{align}
	\mathbf{T}^P (\vec{B})= \vec{0} 
	\end{align} 
	Hence, if $\vec{A}$ is a nilpotent matrix then operator $\mathbf{T}$ is also a nilpotent operator.\\
	Let consider $K = 2$ for which $\vec{A}^K = \vec{0}$ that is :
	\begin{align}
	\vec{A}^2 = \vec{0} \label{2.7}\\
\implies	\vec{A}^3 = \vec{0} \label{2.8}
	\end{align} 
	Now using \eqref{2.7} and \eqref{2.8} in \eqref{2.5}, we get:
	\begin{align}
	\mathbf{T}^3(\vec{B}) = \vec{0}
	\end{align}
	Hence $P$ = 3.
	\section{example}
	Let consider a matrices $\vec{A}$ and, $B$ as :
	\begin{align}
\vec{A} = \myvec{2 & -2 \\ 2 & -2}, \vec{B} = \myvec{1 & 2 \\ 3 & 4}\\
\vec{A}.\vec{B} = \myvec{-4 & -4 \\ -4 & -4} \neq \vec{0}\\
\intertext{Now,}
\implies \vec{A}^2 = \myvec{2 & -2 \\ 2 & -2}\myvec{2 & -2 \\ 2 & -2} =\myvec{0 & 0 \\ 0 & 0}\label{3.3}\\
\implies\vec{A}^3 = \myvec{0 & 0 \\ 0 & 0} \label{3.4}
\intertext{Hence $K$ = 2.}
\intertext{And we have also,}
\vec{A}\vec{B}\vec{A} = \myvec{-16 & 16 \\ -16 & 16} \neq \vec{0}
\end{align}
Hence from \eqref{2.3} we conclude, $P \neq 2$. 	Now putting the value of $\vec{A}^3$ from \eqref{3.4} and  value of $\vec{A}^2$ from \eqref{3.3} in \eqref{2.5} we get,
\begin{align}
\mathbf{T}^3(\vec{B}) = \vec{0}
\end{align}
Hence, $P = 3$	for this example.
	
\end{document}