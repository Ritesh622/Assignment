\documentclass[journal,12pt,twocolumn]{IEEEtran}
\usepackage{setspace}
\usepackage{gensymb}
\singlespacing
\usepackage[cmex10]{amsmath}
\usepackage{amsthm}
\usepackage{mathrsfs}
\usepackage{txfonts}
\usepackage{stfloats}
\usepackage{bm}
\usepackage[table,xcdraw]{xcolor}
\usepackage{cite}
\usepackage{cases}
\usepackage{subfig}
\usepackage{longtable}
\usepackage{multirow}
%\usepackage{algorithm}
\usepackage{enumitem}
\usepackage{mathtools}
\usepackage{steinmetz}
\usepackage{tikz}
\usepackage{circuitikz}
\usepackage{verbatim}
\usepackage{tfrupee}
\usepackage[breaklinks=true]{hyperref}
%\usepackage{stmaryrd}
\usepackage{tkz-euclide} % loads  TikZ and tkz-base
%\usetkzobj{all}
\usetikzlibrary{calc,math}
\usepackage{listings}
\usepackage{color}                                            %%
\usepackage{array}                                            %%
\usepackage{longtable}                                        %%
\usepackage{calc}                                             %%
\usepackage{multirow}                                         %%
\usepackage{hhline}                                           %%
\usepackage{ifthen}                                           %%
%optionally (for landscape tables embedded in another document): %%
\usepackage[]{caption}
\usepackage{lscape}     
\usepackage{multicol}
\usepackage{chngcntr}
\usepackage{enumerate}
%\usepackage{placeins}
%\usepackage{float}
%\restylefloat{table}
%\usepackage{wasysym}
%\newcounter{MYtempeqncnt}
\DeclareMathOperator*{\Res}{Res}
%\renewcommand{\baselinestretch}{2}
\renewcommand\thesection{\arabic{section}}
\renewcommand\thesubsection{\thesection.\arabic{subsection}}
\renewcommand\thesubsubsection{\thesubsection.\arabic{subsubsection}}
\newcommand\numberthis{\addtocounter{equation}{1}\tag{\theequation}}
\renewcommand\thesectiondis{\arabic{section}}
\renewcommand\thesubsectiondis{\thesectiondis.\arabic{subsection}}
\renewcommand\thesubsubsectiondis{\thesubsectiondis.\arabic{subsubsection}}

% correct bad hyphenation here
\hyphenation{op-tical net-works semi-conduc-tor}
\def\inputGnumericTable{}                                 %%

\lstset{
	%language=C,
	frame=single, 
	breaklines=true,
	columns=fullflexible
}


\begin{document}
	%
	
	
	\newtheorem{theorem}{Theorem}[section]
	\newtheorem{problem}{Problem}
	\newtheorem{proposition}{Proposition}[section]
	\newtheorem{lemma}{Lemma}[section]
	\newtheorem{corollary}[theorem]{Corollary}
	\newtheorem{example}{Example}[section]
	\newtheorem{definition}[problem]{Definition}
	
	\newcommand{\BEQA}{\begin{eqnarray}}
	\newcommand{\EEQA}{\end{eqnarray}}
	\newcommand{\define}{\stackrel{\triangle}{=}}
	\bibliographystyle{IEEEtran}
	%\bibliographystyle{ieeetr}
	\providecommand{\mbf}{\mathbf}
	\providecommand{\pr}[1]{\ensuremath{\Pr\left(#1\right)}}
	\providecommand{\qfunc}[1]{\ensuremath{Q\left(#1\right)}}
	\providecommand{\sbrak}[1]{\ensuremath{{}\left[#1\right]}}
	\providecommand{\lsbrak}[1]{\ensuremath{{}\left[#1\right.}}
	\providecommand{\rsbrak}[1]{\ensuremath{{}\left.#1\right]}}
	\providecommand{\brak}[1]{\ensuremath{\left(#1\right)}}
	\providecommand{\lbrak}[1]{\ensuremath{\left(#1\right.}}
	\providecommand{\rbrak}[1]{\ensuremath{\left.#1\right)}}
	\providecommand{\cbrak}[1]{\ensuremath{\left\{#1\right\}}}
	\providecommand{\lcbrak}[1]{\ensuremath{\left\{#1\right.}}
	\providecommand{\rcbrak}[1]{\ensuremath{\left.#1\right\}}}
	\theoremstyle{remark}
	\newtheorem{rem}{Remark}
	\newcommand{\sgn}{\mathop{\mathrm{sgn}}}
	\providecommand{\abs}[1]{\left\vert#1\right\vert}
	\providecommand{\res}[1]{\Res\displaylimits_{#1}} 
	\providecommand{\norm}[1]{\left\lVert#1\right\rVert}
	%\providecommand{\norm}[1]{\lVert#1\rVert}
	\providecommand{\mtx}[1]{\mathbf{#1}}
	\providecommand{\mean}[1]{E\left[ #1 \right]}
	\providecommand{\fourier}{\overset{\mathcal{F}}{ \rightleftharpoons}}
	%\providecommand{\hilbert}{\overset{\mathcal{H}}{ \rightleftharpoons}}
	\providecommand{\system}{\overset{\mathcal{H}}{ \longleftrightarrow}}
	%\newcommand{\solution}[2]{\textbf{Solution:}{#1}}
	\newcommand{\solution}{\noindent \textbf{Solution: }}
	\newcommand{\cosec}{\,\text{cosec}\,}
	\providecommand{\dec}[2]{\ensuremath{\overset{#1}{\underset{#2}{\gtrless}}}}
	\newcommand{\myvec}[1]{\ensuremath{\begin{pmatrix}#1\end{pmatrix}}}
	\newcommand{\mydet}[1]{\ensuremath{\begin{vmatrix}#1\end{vmatrix}}}
	\numberwithin{equation}{subsection}
	\makeatletter
	\@addtoreset{figure}{problem}
	\makeatother
	\let\StandardTheFigure\thefigure
	\let\vec\mathbf
	\renewcommand{\thefigure}{\theproblem}
	\def\putbox#1#2#3{\makebox[0in][l]{\makebox[#1][l]{}\raisebox{\baselineskip}[0in][0in]{\raisebox{#2}[0in][0in]{#3}}}}
	\def\rightbox#1{\makebox[0in][r]{#1}}
	\def\centbox#1{\makebox[0in]{#1}}
	\def\topbox#1{\raisebox{-\baselineskip}[0in][0in]{#1}}
	\def\midbox#1{\raisebox{-0.5\baselineskip}[0in][0in]{#1}}
	\vspace{3cm}
	\title{Matrix Theory Assignment 13}
	\author{Ritesh Kumar \\ EE20RESCH11005}
	
	
	\maketitle
	\newpage
	%\tableofcontents
	\bigskip
	\renewcommand{\thefigure}{\theenumi}
	\renewcommand{\thetable}{\theenumi}
	\counterwithout{figure}{section}
	\counterwithout{figure}{subsection}
	\date{Today}
	

\begin{abstract}
This problem is all about to  introducing the concept of characteristic polynomial over a filed.
\end{abstract}
All the codes for this document can be found at
\begin{lstlisting}
https://github.com/Ritesh622/Assignment_EE5609/tree/master/Assignment_13
\end{lstlisting}
\section{Problem}
Let $A$ be the an $n \times n$ diagonal matrix with characteristic polynomial
\begin{align}
(x-c_1)^{d_1} (x-c_2)^{d_2} \dots (x-c_k)^{d_k} \label{1.1}
\end{align}
Where $c_1, c_2,  \dots c_k $ are distinct. Let $\mathbf{V}$ the space of $n \times n$ matrices $B$ such that 
\begin{align}
AB = BA \label{1.2}
\end{align}
Prove that the dimension of $\mathbf{V}$ is,
\begin{align}
{d_1}^2 + {d_2}^2 \dots + {d_k}^2 \label{1.3}
\end{align}

\section{solution}
Let consider we have a matrix $A$ which is a diagonal matrix, which is  given as 
\begin{align}  
	 A =  \begin{bmatrix}
	c_{1}I & 0        & 0  & \dots & 0     & 0 \\ 
	0      &  c_{2}I  & 0  & \dots & 0     & 0 \\ 
	0      &  0       & 0  & \dots & .     & . \\
	.      &  .       & .  & \dots & .     & .  \\
	0     &  0        & .  & \dots & .     & c_{k}I 
\end{bmatrix}
\end{align}
 Consider $B$ as :
 \begin{align}  
 B =  \begin{bmatrix}
 B_{11}    & B_{12}   & . & \dots   & . & B_{1k}  \\ 
 B_{21}    & B_{22}   & . & \dots   & . & B_{2k} \\ 
 .         &  .       & . & \dots   & . & . \\
 .         &  .       & . & \dots   & . & .  \\
 B_{k1}    & B_{k2}   & . & \dots   & . & B_{kk} 
 \end{bmatrix}
 \end{align} 
Where $B_{ij}$ has dimension $d_{i} \times d_{j}$. Since  we have given ,
\begin{align}
AB = BA
\end{align}
\begin{multline}
\implies   \begin{bmatrix}
c_1B_{11}    & c_1B_{12}   & . & \dots   & . & c_1B_{1k}  \\ 
c_2B_{21}    &c_2 B_{22}   & . & \dots   & . & c_2B_{2k} \\ 
.         &  .       & . & \dots   & . & . \\
.         &  .       & . & \dots   & . & .  \\
c_kB_{k1}    & c_kB_{k2}   & . & \dots   & . & c_kB_{kk} \\
\end{bmatrix} =  \\ \\ \begin{bmatrix}
c_1B_{11}    & B_{12}   & . & \dots   & . & c_1B_{1k}  \\ 
c_2B_{21}    & B_{22}   & . & \dots   & . &c_2 B_{2k} \\ 
.         &  .       & . & \dots   & . & . \\
.         &  .       & . & \dots   & . & .  \\
c_kB_{k1}    & c_kB_{k2}   & . & \dots   & . & c_kB_{kk} \\
\end{bmatrix} \label{2.4}
\end{multline}
 Hence, from above  equation \refeq{2.4} we can conclude,
 \begin{align}
 c_i \neq c_j, \forall i \neq j\\
 \implies B_{ij} = 0, \forall i \neq j
 \end{align}
We can have $B_{11}, B_{22} \dots$ any arbitrary matrices. From \eqref{2.4} we can have  
\begin{align}
 D(B_{ij}) = d_i^{2} 
 \end{align}
Where D  represents dimension of matrix. Therefore the dimension of the space of all such $B_{ij}{'}$s matrices is given as :
\begin{align}
{d_1}^2 + {d_2}^2 \dots + {d_k}^2
\end{align}


\section{Example}

Let suppose we have matrix $A$ as :
\begin{align}  
A =  \begin{bmatrix}
1 & 0  & 0 & 0 \\ 
0 & 1  & 0 & 0  \\
0 & 0  & 2 & 0  \\
0 & 0  & 0 & 2    
\end{bmatrix}
\end{align}

and $B$ as :
\begin{align}  
B =  \begin{bmatrix}
B_{11} & B_{12}\\ 
 B_{21}  & B_{22}  
\end{bmatrix}\\
\intertext{Where}
B_{11} =  \begin{bmatrix}
b_{11} & b_{12}\\ 
b_{21}  & b_{22}  
\end{bmatrix},
B_{12} =  \begin{bmatrix}
b_{13} & b_{14}\\ 
b_{23}  & b_{24}  
\end{bmatrix}\\ \\
B_{21} =  \begin{bmatrix}
b_{31} & b_{32}\\
b_{41}  & b_{42}  
\end{bmatrix},
B_{22} =  \begin{bmatrix}
b_{33} & b_{34}\\ 
b_{43}  & b_{44}  
\end{bmatrix}\\ \\
 \implies B =  \begin{bmatrix}
b_{11} & b_{12}  & b_{13} & b_{14} \\ 
b_{21} & b_{22}  & b_{23} & b_{24}  \\
b_{31} & b_{32}  & b_{33} & b_{34}  \\
b_{41} & b_{42}  & b_{43} & b_{44}    
\end{bmatrix}
\intertext{Consider}
C = AB\\
\implies C = \begin{bmatrix}
b_{11} & b_{12}  & b_{13} & b_{14} \\ 
b_{21} & b_{22}  & b_{23} & b_{24}  \\
2b_{31} & 2b_{32}  & 2b_{33} & 2b_{34}  \\
2b_{41} & 2b_{42}  & 2b_{43} & 2b_{44}    
\end{bmatrix}
\intertext{Let another matrix D as :}
D = BA\\
B =  \begin{bmatrix}
b_{11} & b_{12}  & 2b_{13} & 2b_{14} \\ 
b_{21} & b_{22}  & 2b_{23} & 2b_{24}  \\
b_{31} & b_{32}  & 2b_{33} & 2b_{34}  \\
b_{41} & b_{42}  & 2b_{43} & 2b_{44}    
\end{bmatrix}
\end{align}
We have given as,
\begin{align}
BA = AB\\
\implies C = D
\end{align}
\begin{align}
\intertext{it is possible only when}
b_{13}= b_{14} = b_{23}= b_{24} = 0 \\
\intertext{and}
 b_{31} = b_{32} = b_{41} = b_{42} = 0
 \end{align}
 \begin{align}
B_{12} =  \begin{bmatrix}
b_{13} & b_{14}\\ 
b_{23}  & b_{24}  
\end{bmatrix} =  \begin{bmatrix}
0 & 0\\ 
0  & 0  
\end{bmatrix}\\ 
\intertext{And}
B_{21} =  \begin{bmatrix}
b_{31} & b_{32}\\
b_{41}  & b_{42}  
\end{bmatrix}  =  \begin{bmatrix}
0 & 0\\ 
0  & 0  
\end{bmatrix} 
\end{align}
Hence, therefore matrix $B$ becomes,
\begin{align}
B =  \begin{bmatrix}
b_{11} & b_{12}  & 0       & 0 \\ 
b_{21} & b_{22}  & 0       & 0  \\
0      & 0       & 2b_{33} & 2b_{34}  \\
0      & 0       & 2b_{43} & 2b_{44}    
\end{bmatrix}\\
\implies B =  \begin{bmatrix}
B_{11} & \huge{0} \\ 
\huge{0}  & B_{22}  
\end{bmatrix}\\
\implies B_{ij} = 0, \forall i \neq j
\end{align}

Now the basis of the $n \times n $  matrices for vector space of all nxn matrix B are,
\begin{align}
\beta_1 =  \begin{bmatrix}
1 & 0  & 0 & 0 \\ 
0 & 0  & 0 & 0  \\
0 & 0  & 0 & 0  \\
0 & 0  & 0 & 0    
\end{bmatrix},
 \beta_2 =  \begin{bmatrix}
0 & 1  & 0 & 0 \\ 
0 & 0  & 0 & 0  \\
0 & 0  & 0 & 0  \\
0 & 0  & 0 & 0    
\end{bmatrix}\\
\beta_3 =  \begin{bmatrix}
0 & 0  & 0 & 0 \\ 
1 & 0  & 0 & 0  \\
0 & 0  & 0 & 0  \\
0 & 0  & 0 & 0    
\end{bmatrix},
\beta_4 =  \begin{bmatrix}
0 & 0  & 0 & 0 \\ 
0 & 1  & 0 & 0  \\
0 & 0  & 0 & 0  \\
0 & 0  & 0 & 0    
\end{bmatrix}\\
\beta_5 =  \begin{bmatrix}
0 & 0  & 0 & 0 \\ 
0 & 0  & 0 & 0  \\
0 & 0  & 1 & 0  \\
0 & 0  & 0 & 0    
\end{bmatrix},
\beta_6 =  \begin{bmatrix}
0 & 0  & 0 & 0 \\ 
0 & 0  & 0 & 0  \\
0 & 0  & 0 & 1  \\
0 & 0  & 0 & 0    
\end{bmatrix}\\
\beta_7 =  \begin{bmatrix}
1 & 0  & 0 & 0 \\ 
0 & 0  & 0 & 0  \\
0 & 0  & 0 & 0  \\
0 & 0  & 1 & 0    
\end{bmatrix},
\beta_8 =  \begin{bmatrix}
0 & 0  & 0 & 0 \\ 
0 & 0  & 0 & 0  \\
0 & 0  & 0 & 0  \\
0 & 0  & 0 & 1    
\end{bmatrix},
\end{align}
Thus, Dimension of $\mathbf{V}$ $($vector space of all $n \times n$ matrices $ B ) = 8$
\begin{align}
\intertext{Also}
 {d_1}^{2} + d{_2}^2 = 2^{2}+2^{2} = 8. 
 \end{align}
Therefore, Dimension of $\mathbf{V}$ $($vector space of all $ n \times n$ matrix $B$ is :
\begin{align}
 d_1^2+d_2^2
 \end{align}
\end{document}