
\documentclass[journal,12pt,twocolumn]{IEEEtran}
\usepackage{setspace}
\usepackage{gensymb}
\singlespacing
\usepackage[cmex10]{amsmath}
\usepackage{amsthm}
\usepackage{mathrsfs}
\usepackage{txfonts}
\usepackage{stfloats}
\usepackage{bm}
\usepackage{cite}
\usepackage{cases}
\usepackage{subfig}
\usepackage{float}
\usepackage{longtable}
\usepackage{multirow}
\usepackage{caption}
%\usepackage[font=bf,labelfont=bf]{caption}
\usepackage{enumitem}
\usepackage{mathtools}
\usepackage{steinmetz}
\usepackage{tikz}
\usepackage{circuitikz}
\usepackage{verbatim}
\usepackage{tfrupee}
\usepackage[breaklinks=true]{hyperref}
\usepackage{tkz-euclide}
\usetikzlibrary{calc,math}
\usepackage{listings}
    \usepackage{color}                                            %%
    \usepackage{array}                                            %%
    \usepackage{longtable}                                        %%
    \usepackage{calc}                                             %%
    \usepackage{multirow}                                         %%
    \usepackage{hhline}                                           %%
    \usepackage{ifthen}                                           %%
    \usepackage{lscape}     
\usepackage{multicol}
\usepackage{chngcntr}


\DeclareMathOperator*{\Res}{Res}

\renewcommand\thesection{\arabic{section}}
\renewcommand\thesubsection{\thesection.\arabic{subsection}}
\renewcommand\thesubsubsection{\thesubsection.\arabic{subsubsection}}

\renewcommand\thesectiondis{\arabic{section}}
\renewcommand\thesubsectiondis{\thesectiondis.\arabic{subsection}}
\renewcommand\thesubsubsectiondis{\thesubsectiondis.\arabic{subsubsection}}
\numberwithin{table}{section}

\hyphenation{op-tical net-works semi-conduc-tor}
\def\inputGnumericTable{}                                 %%

\lstset{
%language=C,
frame=single, 
breaklines=true,
columns=fullflexible
}
\begin{document}


\newtheorem{theorem}{Theorem}[section]
\newtheorem{problem}{Problem}
\newtheorem{proposition}{Proposition}[section]
\newtheorem{lemma}{Lemma}[section]
\newtheorem{corollary}[theorem]{Corollary}
\newtheorem{example}{Example}[section]
\newtheorem{definition}[problem]{Definition}

\newcommand{\BEQA}{\begin{eqnarray}}
\newcommand{\EEQA}{\end{eqnarray}}
\newcommand{\define}{\stackrel{\triangle}{=}}
\bibliographystyle{IEEEtran}
\providecommand{\mbf}{\mathbf}
\providecommand{\pr}[1]{\ensuremath{\Pr\left(#1\right)}}
\providecommand{\qfunc}[1]{\ensuremath{Q\left(#1\right)}}
\providecommand{\sbrak}[1]{\ensuremath{{}\left[#1\right]}}
\providecommand{\lsbrak}[1]{\ensuremath{{}\left[#1\right.}}
\providecommand{\rsbrak}[1]{\ensuremath{{}\left.#1\right]}}
\providecommand{\brak}[1]{\ensuremath{\left(#1\right)}}
\providecommand{\lbrak}[1]{\ensuremath{\left(#1\right.}}
\providecommand{\rbrak}[1]{\ensuremath{\left.#1\right)}}
\providecommand{\cbrak}[1]{\ensuremath{\left\{#1\right\}}}
\providecommand{\lcbrak}[1]{\ensuremath{\left\{#1\right.}}
\providecommand{\rcbrak}[1]{\ensuremath{\left.#1\right\}}}
\theoremstyle{remark}
\newtheorem{rem}{Remark}
\newcommand{\sgn}{\mathop{\mathrm{sgn}}}
\providecommand{\abs}[1]{\left\vert#1\right\vert}
\providecommand{\res}[1]{\Res\displaylimits_{#1}} 
\providecommand{\norm}[1]{\left\lVert#1\right\rVert}
%\providecommand{\norm}[1]{\lVert#1\rVert}
\providecommand{\mtx}[1]{\mathbf{#1}}
\providecommand{\mean}[1]{E\left[ #1 \right]}
\providecommand{\fourier}{\overset{\mathcal{F}}{ \rightleftharpoons}}
%\providecommand{\hilbert}{\overset{\mathcal{H}}{ \rightleftharpoons}}
\providecommand{\system}{\overset{\mathcal{H}}{ \longleftrightarrow}}
	%\newcommand{\solution}[2]{\textbf{Solution:}{#1}}
\newcommand{\solution}{\noindent \textbf{Solution: }}
\newcommand{\cosec}{\,\text{cosec}\,}
\providecommand{\dec}[2]{\ensuremath{\overset{#1}{\underset{#2}{\gtrless}}}}
\newcommand{\myvec}[1]{\ensuremath{\begin{pmatrix}#1\end{pmatrix}}}
\newcommand{\mydet}[1]{\ensuremath{\begin{vmatrix}#1\end{vmatrix}}}
\numberwithin{equation}{subsection}
\makeatletter
\@addtoreset{figure}{problem}
\makeatother
\let\StandardTheFigure\thefigure
\let\vec\mathbf
\renewcommand{\thefigure}{\theproblem}
\def\putbox#1#2#3{\makebox[0in][l]{\makebox[#1][l]{}\raisebox{\baselineskip}[0in][0in]{\raisebox{#2}[0in][0in]{#3}}}}
     \def\rightbox#1{\makebox[0in][r]{#1}}
     \def\centbox#1{\makebox[0in]{#1}}
     \def\topbox#1{\raisebox{-\baselineskip}[0in][0in]{#1}}
     \def\midbox#1{\raisebox{-0.5\baselineskip}[0in][0in]{#1}}
\vspace{3cm}
\title{Matrix Theory Assignment 16}
\author{Ritesh Kumar \\ EE20RESCH11005}
\maketitle
\newpage
\bigskip
\renewcommand{\thefigure}{\theenumi}
\renewcommand{\thetable}{\theenumi}
All the codes for this document can be found at
%
\begin{lstlisting}
https://github.com/Ritesh622/Assignment_EE5609/tree/master/Assignment_16
\end{lstlisting}
%
	\section{Problem}
	Consider a matrix,
	\begin{align}
	\vec{A} = \myvec{2 & 2 & 1 \\0 & 2 & -1 \\ 0 & 0 & 3}\\ \intertext{and,} \vec{B} = \myvec{2 & 1 & 0 \\ 0 & 2 & 0 \\ 0 & 0 & 3}
	\end{align}
	
Then which of following is true,
\begin{enumerate}
\item $\vec{A}$ and $\vec{B}$ is similar over the field of rational numbers.
\item $\vec{A}$ is diagonalizable over the field of rational numbers $\mathbb{Q}$.
\item $\vec{B}$ is the Jordan canonical form of $\vec{A}$.
\item The minimal polynomial and the characteristic polynomial of $\vec{A}$ are the same.
\end{enumerate}
	\section{solution}
	\subsection{Part 1}
Two matrix are said to be similar if their eigen values are same.\\ Eigen value of $\vec{A}$ is given as:
\begin{align}
 \myvec{2 - \lambda & 2 & 1 \\0 & 2-\lambda & -1 \\ 0 & 0 & 3-\lambda} = 0\\
 \implies -{\lambda}^3+7{\lambda}^2 -16\lambda +12 = 0\\
\implies \lambda_1  = 2, \lambda_2 = 2, \lambda_3 = 3. \label{2.3}\\
 \intertext{Similarally, eigen values of $\vec{B}$ is givem as:}
\myvec{2 - \lambda & 1 0 \\ 0 & 2 - \lambda & 0 \\ 0 & 0 & 3 - \lambda}\\
\implies -{\lambda}^3+7{\lambda}^2 -16\lambda +12 = 0\\
\implies \lambda_1  = 2, \lambda_2 = 2, \lambda_3 = 3.
\end{align}
Hence, matrices  $\vec{A}$ and $\vec{B}$ are similar.
\subsection{Part B}
Matrix $\vec{A}$ is diagonalizable if and only if there is a basis of $\mathbb{R}^3$ consisting of eigenvectors of $\vec{A}$.

From \eqref{2.3} , our eigenvalues for $\vec{A}$ are,
\begin{align}
\lambda_1 = \lambda_2 = 2 \intertext{and,} \lambda_3 = 3 .
\end{align}
 Hence  $\lambda_1 = \lambda_2 $ is a repeated root with multiplicity two. Hence, We can get only two linearly independent eigenvectors for $\vec{A,}$ are given  as :
 \begin{align}
 \myvec{1 \\ 0 \\ 0} \textit{and,} \myvec{-1 \\ -1 \\ 1}
 \end{align}
But any basis for $\mathbb{R}^3$ consists of three vectors. Therefore there is no third eigenbasis for $\vec{A}$, hence $\vec{A}$ is not diagonalizable.
\subsection{part 3}
From \eqref{2.3} we have eigenvalue $\lambda_1 = 2 $ with geometic multiplicity 2. Hence the Jordon canonical form of $\vec{A}$ can be written as :
\begin{align}
\vec{J}_\vec{A} = \myvec{2 & 1 & 0 \\ 0 & 2 & 0 \\ 0 & 0 & 3}
\end{align}
Hence $\vec{B}$ is the Jordan canonical form of $\vec{A}$.
 \subsection{Part 4}
 From \eqref{2.3}, the characteristic polynomial of this matrix is: 
 \begin{align}
f(\lambda) = -{\lambda}^3+7{\lambda}^2 -16\lambda +12 = (\lambda - 2)^2 (\lambda - 3)
 \end{align}
 Minimal polynomial for a matrix is a smallest polynomial for which
 \begin{align}
 M_{\vec{A}}(x) = 0 \label{2.4.1}
 \end{align}
 Using \eqref{2.4.1}, we found minimal polynomial of $\vec{A}$ is :
 \begin{align}
 M_{\vec{A}}(x) = (x-2)^2(x-3) \label{2.4.2}
 \end{align}	
 We can relate the minimal polynomial with the size of Jordan block.\\\\
 
\textbf{Size of Jordan block $=$ degree of minimal polynomial with geometic multiplicity of the eigen values.}\\\\
From \eqref{2.4.2} we can observe that, geometric multiplicity of eigen value 2 is 2. Hence size of Jordan block is 2. which is given as:
\begin{align}
\myvec{2 & 1 \\ 0 & 2 }
\end{align}
if geometric multiplicity of $\lambda = 2$ would be 3, then Jordan block would be:
\begin{align}
\myvec{2 & 1 & 0 \\ 0 & 2 & 1 \\ 0 & 0 &2}
\end{align}
In \eqref{2.4.2} geometric multiplicity of eigen value 2 is 2, and geometric multiplicity of eigen value 3 is one hence jardon block is:
\begin{align}
\myvec{2 & 1 & 0 \\ 0 & 2 & 0 \\ 0 & 0 & 3}
\end{align}

\end{document}