 \documentclass[journal,12pt,twocolumn]{IEEEtran}
%
\usepackage{setspace}
\usepackage{gensymb}
\singlespacing
\usepackage[cmex10]{amsmath}
\usepackage{amsthm}
\usepackage{mathrsfs}
\usepackage{txfonts}
\usepackage{stfloats}
\usepackage{bm}
\usepackage{cite}
\usepackage{cases}
\usepackage{subfig}
\usepackage{longtable}
\usepackage{multirow}
%\usepackage{algorithm}
\usepackage{enumitem}
\usepackage{mathtools}
\usepackage{steinmetz}
\usepackage{tikz}
\usepackage{circuitikz}
\usepackage{verbatim}
\usepackage{tfrupee}
\usepackage[breaklinks=true]{hyperref}
%\usepackage{stmaryrd}
\usepackage{tkz-euclide} % loads  TikZ and tkz-base
%\usetkzobj{all}
\usetikzlibrary{calc,math}
\usepackage{listings}
\usepackage{color}                                            %%
\usepackage{array}                                            %%
\usepackage{longtable}                                        %%
\usepackage{calc}                                             %%
\usepackage{multirow}                                         %%
\usepackage{hhline}                                           %%
\usepackage{ifthen}                                           %%
%optionally (for landscape tables embedded in another document): %%
\usepackage{lscape}     
\usepackage{multicol}
\usepackage{chngcntr}
%\usepackage{enumerate}

%\usepackage{wasysym}
%\newcounter{MYtempeqncnt}
\DeclareMathOperator*{\Res}{Res}
%\renewcommand{\baselinestretch}{2}
\renewcommand\thesection{\arabic{section}}
\renewcommand\thesubsection{\thesection.\arabic{subsection}}
\renewcommand\thesubsubsection{\thesubsection.\arabic{subsubsection}}
\newcommand\numberthis{\addtocounter{equation}{1}\tag{\theequation}}
\renewcommand\thesectiondis{\arabic{section}}
\renewcommand\thesubsectiondis{\thesectiondis.\arabic{subsection}}
\renewcommand\thesubsubsectiondis{\thesubsectiondis.\arabic{subsubsection}}

% correct bad hyphenation here
\hyphenation{op-tical net-works semi-conduc-tor}
\def\inputGnumericTable{}                                 %%

\lstset{
	%language=C,
	frame=single, 
	breaklines=true,
	columns=fullflexible
}


\begin{document}
	%
	
	
	\newtheorem{theorem}{Theorem}[section]
	\newtheorem{problem}{Problem}
	\newtheorem{proposition}{Proposition}[section]
	\newtheorem{lemma}{Lemma}[section]
	\newtheorem{corollary}[theorem]{Corollary}
	\newtheorem{example}{Example}[section]
	\newtheorem{definition}[problem]{Definition}
	
	\newcommand{\BEQA}{\begin{eqnarray}}
	\newcommand{\EEQA}{\end{eqnarray}}
	\newcommand{\define}{\stackrel{\triangle}{=}}
	\bibliographystyle{IEEEtran}
	%\bibliographystyle{ieeetr}
	\providecommand{\mbf}{\mathbf}
	\providecommand{\pr}[1]{\ensuremath{\Pr\left(#1\right)}}
	\providecommand{\qfunc}[1]{\ensuremath{Q\left(#1\right)}}
	\providecommand{\sbrak}[1]{\ensuremath{{}\left[#1\right]}}
	\providecommand{\lsbrak}[1]{\ensuremath{{}\left[#1\right.}}
	\providecommand{\rsbrak}[1]{\ensuremath{{}\left.#1\right]}}
	\providecommand{\brak}[1]{\ensuremath{\left(#1\right)}}
	\providecommand{\lbrak}[1]{\ensuremath{\left(#1\right.}}
	\providecommand{\rbrak}[1]{\ensuremath{\left.#1\right)}}
	\providecommand{\cbrak}[1]{\ensuremath{\left\{#1\right\}}}
	\providecommand{\lcbrak}[1]{\ensuremath{\left\{#1\right.}}
	\providecommand{\rcbrak}[1]{\ensuremath{\left.#1\right\}}}
	\theoremstyle{remark}
	\newtheorem{rem}{Remark}
	\newcommand{\sgn}{\mathop{\mathrm{sgn}}}
	\providecommand{\abs}[1]{\left\vert#1\right\vert}
	\providecommand{\res}[1]{\Res\displaylimits_{#1}} 
	\providecommand{\norm}[1]{\left\lVert#1\right\rVert}
	%\providecommand{\norm}[1]{\lVert#1\rVert}
	\providecommand{\mtx}[1]{\mathbf{#1}}
	\providecommand{\mean}[1]{E\left[ #1 \right]}
	\providecommand{\fourier}{\overset{\mathcal{F}}{ \rightleftharpoons}}
	%\providecommand{\hilbert}{\overset{\mathcal{H}}{ \rightleftharpoons}}
	\providecommand{\system}{\overset{\mathcal{H}}{ \longleftrightarrow}}
	%\newcommand{\solution}[2]{\textbf{Solution:}{#1}}
	\newcommand{\solution}{\noindent \textbf{Solution: }}
	\newcommand{\cosec}{\,\text{cosec}\,}
	\providecommand{\dec}[2]{\ensuremath{\overset{#1}{\underset{#2}{\gtrless}}}}
	\newcommand{\myvec}[1]{\ensuremath{\begin{pmatrix}#1\end{pmatrix}}}
	\newcommand{\mydet}[1]{\ensuremath{\begin{vmatrix}#1\end{vmatrix}}}
	\numberwithin{equation}{subsection}
	\makeatletter
	\@addtoreset{figure}{problem}
	\makeatother
	\let\StandardTheFigure\thefigure
	\let\vec\mathbf
	\renewcommand{\thefigure}{\theproblem}
	\def\putbox#1#2#3{\makebox[0in][l]{\makebox[#1][l]{}\raisebox{\baselineskip}[0in][0in]{\raisebox{#2}[0in][0in]{#3}}}}
	\def\rightbox#1{\makebox[0in][r]{#1}}
	\def\centbox#1{\makebox[0in]{#1}}
	\def\topbox#1{\raisebox{-\baselineskip}[0in][0in]{#1}}
	\def\midbox#1{\raisebox{-0.5\baselineskip}[0in][0in]{#1}}
	\vspace{3cm}
	\title{Matrix Theory Assignment 7}
	\author{Ritesh Kumar \\ EE20RESCH11005}
	
	
	\maketitle
	\newpage
	%\tableofcontents
	\bigskip
	\renewcommand{\thefigure}{\theenumi}
	\renewcommand{\thetable}{\theenumi}
	\counterwithout{figure}{section}
	\counterwithout{figure}{subsection}
	%\begin{document}	
	%	\begin{titlepage}
	%	\begin{center}
	%		\vspace*{1cm}
	%		
	%		\textbf{ \huge{Assignment 1}}
	%		\vspace{1.5cm}
	%		
	%		\textbf{Ritesh Kumar} \\
	%		textbf{(EE20RESCH11005)}\\
	%		\textbf{Communication and Signal Processing}
	\date{Today}
	
	%	\end{center}
	%	\end{titlepage}
\begin{abstract}
This problem demonstrate a method to  find the solution of the given system of equation using linear algebra.
\end{abstract}
All the codes for the figure in this document can be found at
\begin{lstlisting}
https://github.com/Ritesh622/Assignment_EE5609/tree/master/Assignment_7
\end{lstlisting}
\section{\textbf{Problem}}
Consider the system of the equations
\begin{align}
x_1 - x_2 +2x_3 = 1  \label{1.1} \\
x_1 - 0x_2 + 2x_3 = 1 \label{1.2} \\
x_1 -3x_2 + 4x_3 = 2 \label{1.3}
\end{align}

Does this system have a solution $?$  If   so describe explicitly all solutions.

\section{Solution}
Let $\vec{V}$ is the set of all  $\myvec{x_1, & x_2, & x_3 }\in \mathbb{R}^{3} $ which satisfy the \eqref{1.1}, \eqref{1.2} and \eqref{1.3}\\
From equation \eqref{1.1} to \eqref{1.3} we can write,
\begin{align}
\myvec{1 & -1 & 2 \\ 1 & 0 & 2 \\ 1 & -3 & 4}\vec{x}= \myvec{1 \\ 1 \\ 2}\\
\implies \vec{A}\vec{x} = \vec{b}  \label{2.2}\\
\intertext{Where,}\\
\vec{A} = \myvec{1 & -1 & 2 \\ 1 & 0 & 2 \\ 1 & -3 & 4}, \vec{b}= \myvec{1 \\ 1 \\ 2}, \vec{x} = \myvec{x_1 \\ x_2 \\ x_3} \label{2.4}
\end{align}
Solving the  matrix $\vec{A}$ for rank  we get,
\begin{align}
\myvec{1&-1& 2\\ 2& 0 & 2 \\ 1 & -3 & 4} &\xleftrightarrow{R_2=R_1 - 2R_1}\myvec{1&-1&2 \\ 0 &2 & -2 \\ 1 &-3 & 4}\\
&\xleftrightarrow{R_3=R_3 -R_1}\myvec{1 &-1&2 \\0 &2 &-2 \\ 0 & -2 & 2}\\
&\xleftrightarrow{R_3=R_3 + R_2}\myvec{1 &-1&2 \\0 &2 &-2 \\ 0 & 0 & 0}
\end{align}
Hence, rank $\left(  \vec{A}  \right) = 2. $
Now solving the augmented matrix of \eqref{2.2} we get,
\begin{align}
\myvec{1&-1& 2 & 1 \\ 2& 0 & 2 & 1 \\ 1 & -3 & 4 & 2 } &\xleftrightarrow{R_2=R_1 - 2R_1}\myvec{1&-1&2 & 1  \\ 0 &2 & -2 & -1 \\ 1 &-3 & 4 & 2}\\
&\xleftrightarrow{R_3=R_3 -R_1}\myvec{1 &-1&2 & 1\\0 &2 &-2 &-1 \\ 0 & -2 & 2 & 1} \\
&\xleftrightarrow{R_3=R_3 + R_2}\myvec{1 &-1&2 & 1 \\0 &2 &-2 & -1 \\ 0 & 0 & 0 & 0}
\end{align}
We have rank $\left( \vec{A} \right) = $  rank $\left( \vec{A:b}\right) = 2 < n $, where $n = 3. $ Hence we have infinite no of solutions for given system of equations.\\
Using Gauss - Jordan elimination method to getting the solution,
\begin{align}
\myvec{1&-1& 2 & 1 \\ 2& 0 & 2 & 1 \\ 1 & -3 & 4 & 2 } &\xleftrightarrow{R_2=R_1 - 2R_1}\myvec{1&-1&2 & 1  \\ 0 &2 & -2 & -1 \\ 1 &-3 & 4 & 2}
\end{align}
\begin{align}
&\xleftrightarrow{R_3=R_3 -R_1}\myvec{1 &-1&2 & 1\\0 &2 &-2 &-1 \\ 0 & -2 & 2 & 1} \\
&\xleftrightarrow{R_2= \frac{R_2}{2}}\myvec{1 &-1&2 & 1 \\ 0 & 1 & -1 & -\frac{1}{2} \\ 0 & -2 & 2 & 1} 
\end{align}
\begin{align}
&\xleftrightarrow{R_3 = R_3 +2R_2}\myvec{1 &-1&2 & 1 \\ 0 & 1 & -1 & -\frac{1}{2} \\ 0 & 0 & 0 & 0} \\ 
&\xleftrightarrow {R_1 = R_1 + R_2}\myvec{1 & 0 & 1 & \frac{1}{2} \\ 0 & 1 & -1 & -\frac{1}{2} \\ 0 & 0 & 0 & 0}
\end{align}
\begin{align}
\implies x_1 +x_3 = \frac{1}{2}, x_2 - x_3 = - \frac{1}{2} \label{2.16}\\
\implies x_2 = - \frac{1}{2} +x_3 , x_1 = \frac{1}{2} - x_3 \label{2.17}
\end{align}
From  equation \eqref{2.16} and \eqref{2.17}
\begin{align}
\vec{x} = \myvec{\frac{1}{2} - x_3 \\ -\frac{1}{2} + x_3 \\ x_3 }
\end{align}

which can be written as,
\begin{align}
\vec{x}=x_3\myvec{-1 \\1\\1} + \myvec{-\frac{1}{2} \\ -\frac{1}{2}\\ 0} \label{2.19}
\end{align}

from \ref{2.19} we can say that for any value $x_3$, $\vec{V}$ will no be gives zero vector. Hence the given solution space will not span of  the vector space $\vec{V}$



\end{document}