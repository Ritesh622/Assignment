
\documentclass[journal,12pt,twocolumn]{IEEEtran}
\usepackage{setspace}
\usepackage{gensymb}
\singlespacing
\usepackage[cmex10]{amsmath}
\usepackage{amsthm}
\usepackage{mathrsfs}
\usepackage{txfonts}
\usepackage{stfloats}
\usepackage{bm}
\usepackage{cite}
\usepackage{cases}
\usepackage{subfig}
\usepackage{float}
\usepackage{longtable}
\usepackage{multirow}
\usepackage{caption}
\usepackage{enumitem}
\usepackage{mathtools}
\usepackage{steinmetz}
\usepackage{tikz}
\usepackage{circuitikz}
\usepackage{verbatim}
\usepackage{tfrupee}
\usepackage[breaklinks=true]{hyperref}
\usepackage{tkz-euclide}
\usetikzlibrary{calc,math}
\usepackage{listings}
    \usepackage{color}                                            %%
    \usepackage{array}                                            %%
    \usepackage{longtable}                                        %%
    \usepackage{calc}                                             %%
    \usepackage{multirow}                                         %%
    \usepackage{hhline}                                           %%
    \usepackage{ifthen}                                           %%
    \usepackage{lscape}     
\usepackage{multicol}
\usepackage{chngcntr}


\DeclareMathOperator*{\Res}{Res}

\renewcommand\thesection{\arabic{section}}
\renewcommand\thesubsection{\thesection.\arabic{subsection}}
\renewcommand\thesubsubsection{\thesubsection.\arabic{subsubsection}}

\renewcommand\thesectiondis{\arabic{section}}
\renewcommand\thesubsectiondis{\thesectiondis.\arabic{subsection}}
\renewcommand\thesubsubsectiondis{\thesubsectiondis.\arabic{subsubsection}}
\numberwithin{table}{section}

\hyphenation{op-tical net-works semi-conduc-tor}
\def\inputGnumericTable{}                                 %%

\lstset{
%language=C,
frame=single, 
breaklines=true,
columns=fullflexible
}
\begin{document}


\newtheorem{theorem}{Theorem}[section]
\newtheorem{problem}{Problem}
\newtheorem{proposition}{Proposition}[section]
\newtheorem{lemma}{Lemma}[section]
\newtheorem{corollary}[theorem]{Corollary}
\newtheorem{example}{Example}[section]
\newtheorem{definition}[problem]{Definition}

\newcommand{\BEQA}{\begin{eqnarray}}
\newcommand{\EEQA}{\end{eqnarray}}
\newcommand{\define}{\stackrel{\triangle}{=}}
\bibliographystyle{IEEEtran}
\providecommand{\mbf}{\mathbf}
\providecommand{\pr}[1]{\ensuremath{\Pr\left(#1\right)}}
\providecommand{\qfunc}[1]{\ensuremath{Q\left(#1\right)}}
\providecommand{\sbrak}[1]{\ensuremath{{}\left[#1\right]}}
\providecommand{\lsbrak}[1]{\ensuremath{{}\left[#1\right.}}
\providecommand{\rsbrak}[1]{\ensuremath{{}\left.#1\right]}}
\providecommand{\brak}[1]{\ensuremath{\left(#1\right)}}
\providecommand{\lbrak}[1]{\ensuremath{\left(#1\right.}}
\providecommand{\rbrak}[1]{\ensuremath{\left.#1\right)}}
\providecommand{\cbrak}[1]{\ensuremath{\left\{#1\right\}}}
\providecommand{\lcbrak}[1]{\ensuremath{\left\{#1\right.}}
\providecommand{\rcbrak}[1]{\ensuremath{\left.#1\right\}}}
\theoremstyle{remark}
\newtheorem{rem}{Remark}
\newcommand{\sgn}{\mathop{\mathrm{sgn}}}
\providecommand{\abs}[1]{\left\vert#1\right\vert}
\providecommand{\res}[1]{\Res\displaylimits_{#1}} 
\providecommand{\norm}[1]{\left\lVert#1\right\rVert}
%\providecommand{\norm}[1]{\lVert#1\rVert}
\providecommand{\mtx}[1]{\mathbf{#1}}
\providecommand{\mean}[1]{E\left[ #1 \right]}
\providecommand{\fourier}{\overset{\mathcal{F}}{ \rightleftharpoons}}
%\providecommand{\hilbert}{\overset{\mathcal{H}}{ \rightleftharpoons}}
\providecommand{\system}{\overset{\mathcal{H}}{ \longleftrightarrow}}
	%\newcommand{\solution}[2]{\textbf{Solution:}{#1}}
\newcommand{\solution}{\noindent \textbf{Solution: }}
\newcommand{\cosec}{\,\text{cosec}\,}
\providecommand{\dec}[2]{\ensuremath{\overset{#1}{\underset{#2}{\gtrless}}}}
\newcommand{\myvec}[1]{\ensuremath{\begin{pmatrix}#1\end{pmatrix}}}
\newcommand{\mydet}[1]{\ensuremath{\begin{vmatrix}#1\end{vmatrix}}}
\numberwithin{equation}{subsection}
\makeatletter
\@addtoreset{figure}{problem}
\makeatother
\let\StandardTheFigure\thefigure
\let\vec\mathbf
\renewcommand{\thefigure}{\theproblem}
\def\putbox#1#2#3{\makebox[0in][l]{\makebox[#1][l]{}\raisebox{\baselineskip}[0in][0in]{\raisebox{#2}[0in][0in]{#3}}}}
     \def\rightbox#1{\makebox[0in][r]{#1}}
     \def\centbox#1{\makebox[0in]{#1}}
     \def\topbox#1{\raisebox{-\baselineskip}[0in][0in]{#1}}
     \def\midbox#1{\raisebox{-0.5\baselineskip}[0in][0in]{#1}}
\vspace{3cm}
\title{Matrix Theory Assignment 17}
\author{Ritesh Kumar \\ EE20RESCH11005}
\maketitle
\newpage
\bigskip
\renewcommand{\thefigure}{\theenumi}
\renewcommand{\thetable}{\theenumi}
All the codes for this document can be found at
%
\begin{lstlisting}
https://github.com/Ritesh622/Assignment_EE5609/tree/master/Assignment_17
\end{lstlisting}
%
	\section{Problem}
Let $f : \mathbb{R}^n  \rightarrow \mathbb{R}^n $ be a continuous  function. such that
\begin{align}
\int_{\mathbb{R}^n} \left | f(x)dx \right | < \infty
\end{align}
Let $A$ be a real $n  \times n$ invertible matrix and  for $x,y \in \mathbb{R}^n$ . Let $<x,y>$ denotes the standard inner product in $\mathbb{R}^n$ then,
$\int_{\mathbb{R}^n} f(Ax) e^{i<y,x>} dx  = ?$ \\



\begin{enumerate}
\item$\int_{\mathbb{R}^n} f(x) e^{i<(A^{-1})^{T}y,x>} \frac{dx}{\left | \text{det} (A) \right |} $\\
\item $\int_{\mathbb{R}^n} f(x) e^{i<A^{T}y,x>} \frac{dx}{\left | \text{det} (A) \right |} $.\\
\item$\int_{\mathbb{R}^n} f(x) e^{i<(A^{T})^{-1}y,x>}dx $.\\
\item $\int_{\mathbb{R}^n} f(x) e^{i<A^{-1}y,x>} \frac{dx}{\left | \text{det} (A) \right |} $.\\
\end{enumerate}
	\section{solution}
Let consider,
\begin{align}
Ax = t \label{2.1}\\
\implies x = A^{-1}t \label{2.2}\\
\implies dx = \frac{dx}{ \left | \text{det} (A) \right |} \label{2.3}
\end{align}
Using  \eqref{2.1} to \eqref{2.3}, We can write: 
\begin{multline}
\int_{\mathbb{R}^n} f(Ax) e^{i<y,x>} dx  = \int_{\mathbb{R}^n} f(t) e^{i<y,(A^{-1}t)>} \frac{dt}{\left| \text{det}(A) \right|} \label{2.5}
\end{multline}
We know that,
\begin{align}
<x,y > = x^Ty = y^Tx\\
\implies <y, (A^{-1}t)> = (y^TA^{-1}t) \label{2.7}\\
\intertext{And}
\implies <(A^{-1})^{T}y,t> = ((A^{-1})^{T}y)^Tt
\end{align}
\begin{multline}
\implies   ((A^{-1})^Ty)^Tt = (y^T((A^{-1})^T)^Tt) = (y^TA^{-1}t) \label{2.9}
\end{multline}
Hence, from \eqref{2.7} and \eqref{2.9}

\begin{align}
<y, (A^{-1}t)>   = <(A^{-1})^{T}y,t>    \label{2.10}
\end{align}
Using \eqref{2.10} in \eqref{2.5} We can write,
\begin{align}
\implies \int_{\mathbb{R}^n} f(t) e^{i<(A^{-1})^{T}y,t>} \frac{dt}{\left| \text{det}(A) \right|}
\end{align}
Hence option $1$ is correct.

\end{document}