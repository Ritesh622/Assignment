
\documentclass[journal,12pt,twocolumn]{IEEEtran}
\usepackage{setspace}
\usepackage{gensymb}
\singlespacing
\usepackage[cmex10]{amsmath}
\usepackage{amsthm}
\usepackage{mathrsfs}
\usepackage{txfonts}
\usepackage{stfloats}
\usepackage{bm}
\usepackage{cite}
\usepackage{cases}
\usepackage{subfig}
\usepackage{float}
\usepackage{longtable}
\usepackage{multirow}
\usepackage{caption}
\usepackage{enumitem}
\usepackage{mathtools}
\usepackage{steinmetz}
\usepackage{tikz}
\usepackage{circuitikz}
\usepackage{verbatim}
\usepackage{tfrupee}
\usepackage[breaklinks=true]{hyperref}
\usepackage{tkz-euclide}
\usetikzlibrary{calc,math}
\usepackage{listings}
    \usepackage{color}                                            %%
    \usepackage{array}                                            %%
    \usepackage{longtable}                                        %%
    \usepackage{calc}                                             %%
    \usepackage{multirow}                                         %%
    \usepackage{hhline}                                           %%
    \usepackage{ifthen}                                           %%
    \usepackage{lscape}     
\usepackage{multicol}
\usepackage{chngcntr}


\DeclareMathOperator*{\Res}{Res}

\renewcommand\thesection{\arabic{section}}
\renewcommand\thesubsection{\thesection.\arabic{subsection}}
\renewcommand\thesubsubsection{\thesubsection.\arabic{subsubsection}}

\renewcommand\thesectiondis{\arabic{section}}
\renewcommand\thesubsectiondis{\thesectiondis.\arabic{subsection}}
\renewcommand\thesubsubsectiondis{\thesubsectiondis.\arabic{subsubsection}}
\numberwithin{table}{section}

\hyphenation{op-tical net-works semi-conduc-tor}
\def\inputGnumericTable{}                                 %%

\lstset{
%language=C,
frame=single, 
breaklines=true,
columns=fullflexible
}
\begin{document}


\newtheorem{theorem}{Theorem}[section]
\newtheorem{problem}{Problem}
\newtheorem{proposition}{Proposition}[section]
\newtheorem{lemma}{Lemma}[section]
\newtheorem{corollary}[theorem]{Corollary}
\newtheorem{example}{Example}[section]
\newtheorem{definition}[problem]{Definition}

\newcommand{\BEQA}{\begin{eqnarray}}
\newcommand{\EEQA}{\end{eqnarray}}
\newcommand{\define}{\stackrel{\triangle}{=}}
\bibliographystyle{IEEEtran}
\providecommand{\mbf}{\mathbf}
\providecommand{\pr}[1]{\ensuremath{\Pr\left(#1\right)}}
\providecommand{\qfunc}[1]{\ensuremath{Q\left(#1\right)}}
\providecommand{\sbrak}[1]{\ensuremath{{}\left[#1\right]}}
\providecommand{\lsbrak}[1]{\ensuremath{{}\left[#1\right.}}
\providecommand{\rsbrak}[1]{\ensuremath{{}\left.#1\right]}}
\providecommand{\brak}[1]{\ensuremath{\left(#1\right)}}
\providecommand{\lbrak}[1]{\ensuremath{\left(#1\right.}}
\providecommand{\rbrak}[1]{\ensuremath{\left.#1\right)}}
\providecommand{\cbrak}[1]{\ensuremath{\left\{#1\right\}}}
\providecommand{\lcbrak}[1]{\ensuremath{\left\{#1\right.}}
\providecommand{\rcbrak}[1]{\ensuremath{\left.#1\right\}}}
\theoremstyle{remark}
\newtheorem{rem}{Remark}
\newcommand{\sgn}{\mathop{\mathrm{sgn}}}
\providecommand{\abs}[1]{\left\vert#1\right\vert}
\providecommand{\res}[1]{\Res\displaylimits_{#1}} 
\providecommand{\norm}[1]{\left\lVert#1\right\rVert}
%\providecommand{\norm}[1]{\lVert#1\rVert}
\providecommand{\mtx}[1]{\mathbf{#1}}
\providecommand{\mean}[1]{E\left[ #1 \right]}
\providecommand{\fourier}{\overset{\mathcal{F}}{ \rightleftharpoons}}
%\providecommand{\hilbert}{\overset{\mathcal{H}}{ \rightleftharpoons}}
\providecommand{\system}{\overset{\mathcal{H}}{ \longleftrightarrow}}
	%\newcommand{\solution}[2]{\textbf{Solution:}{#1}}
\newcommand{\solution}{\noindent \textbf{Solution: }}
\newcommand{\cosec}{\,\text{cosec}\,}
\providecommand{\dec}[2]{\ensuremath{\overset{#1}{\underset{#2}{\gtrless}}}}
\newcommand{\myvec}[1]{\ensuremath{\begin{pmatrix}#1\end{pmatrix}}}
\newcommand{\mydet}[1]{\ensuremath{\begin{vmatrix}#1\end{vmatrix}}}
\numberwithin{equation}{subsection}
\makeatletter
\@addtoreset{figure}{problem}
\makeatother
\let\StandardTheFigure\thefigure
\let\vec\mathbf
\renewcommand{\thefigure}{\theproblem}
\def\putbox#1#2#3{\makebox[0in][l]{\makebox[#1][l]{}\raisebox{\baselineskip}[0in][0in]{\raisebox{#2}[0in][0in]{#3}}}}
     \def\rightbox#1{\makebox[0in][r]{#1}}
     \def\centbox#1{\makebox[0in]{#1}}
     \def\topbox#1{\raisebox{-\baselineskip}[0in][0in]{#1}}
     \def\midbox#1{\raisebox{-0.5\baselineskip}[0in][0in]{#1}}
\vspace{3cm}
\title{Matrix Theory Assignment 17}
\author{Ritesh Kumar \\ EE20RESCH11005}
\maketitle
\newpage
\bigskip
\renewcommand{\thefigure}{\theenumi}
\renewcommand{\thetable}{\theenumi}
All the codes for this document can be found at
%
\begin{lstlisting}
https://github.com/Ritesh622/Assignment_EE5609/tree/master/Assignment_17
\end{lstlisting}
%
	\section{Problem}
Let $f : \mathbb{R}^n  \rightarrow \mathbb{R}^n $ be a continuous  function. such that
\begin{align}
\int_{\mathbb{R}^n} \left | f(\vec{x})d\vec{x} \right | < \infty
\end{align}
Let $A$ be a real $n  \times n$ invertible matrix and  for $x,y \in \mathbb{R}^n$ . Let $ \langle \vec{x},\vec{y} \rangle $ denotes the standard inner product in $\mathbb{R}^n$ then,
$\int_{\mathbb{R}^n} f(\vec{A}\vec{x}) e^{i \langle \vec{y},\vec{x} \rangle} d\vec{x}  = ?$ \\



\begin{enumerate}
\item $\int_{\mathbb{R}^n} f(\vec{x}) e^{i\langle (\vec{A}^{-1})^{T}\vec{y},\vec{x} \rangle} \frac{d\vec{x}}{\left | \text{det} (\vec{A}) \right |} $\\
\item $\int_{\mathbb{R}^n} f(\vec{x}) e^{i \langle \vec{A}^{T}\vec{y},\vec{x} \rangle } \frac{d\vec{x}}{\left | \text{det} (\vec{A}) \right |} $.\\
\item$\int_{\mathbb{R}^n} f(\vec{x}) e^{i \langle (\vec{A}^{T})^{-1}\vec{y},\vec{x} \rangle }d\vec{x} $.\\
\item $\int_{\mathbb{R}^n} f(\vec{x}) e^{i \langle \vec{A}^{-1}\vec{y},\vec{x} \rangle} \frac{d\vec{x}}{\left | \text{det} (\vec{A}) \right |} $.\\
\end{enumerate}
	\section{solution}
Let consider,
\begin{align}
\vec{A}\vec{x} = \vec{t} \label{2.1}\\
\implies \vec{x} = \vec{A}^{-1}\vec{t} \label{2.2}\\
\implies d\vec{x} = \frac{d\vec{t}}{ \left | \text{det} (\vec{A}) \right |} \label{2.3}
\end{align}
Using  \eqref{2.1} to \eqref{2.3}, We can write: 
\begin{multline}
\int_{\mathbb{R}^n} f(\vec{A}\vec{x}) e^{i\langle \vec{y},\vec{x}\rangle} d\vec{x}  = \int_{\mathbb{R}^n} f(\vec{t}) e^{i \langle \vec{y},(\vec{A}^{-1}\vec{t})\rangle} \frac{d\vec{t}}{\left| \text{det}(\vec{A}) \right|} \label{2.5}
\end{multline}
We know that,
\begin{align}
\langle \vec{x},\vec{y} \rangle = \vec{x}^T\vec{y} = \vec{y}^T\vec{x}\\
\implies \langle \vec{y}, (\vec{A}^{-1}\vec{t})\rangle = (\vec{y}^T\vec{A}^{-1}\vec{t}) \label{2.7}\\
\intertext{And}
\implies  \langle (\vec{A}^{-1})^{T}\vec{y},\vec{t}\rangle = ((\vec{A}^{-1})^{T}\vec{y})^T\vec{t}
\end{align}
\begin{multline}
\implies   ((\vec{A}^{-1})^T\vec{y})^T\vec{t} = (\vec{y}^T((\vec{A}^{-1})^T)^T\vec{t}) = (\vec{y}^T\vec{A}^{-1}\vec{t}) \label{2.9}
\end{multline}
Hence, from \eqref{2.7} and \eqref{2.9}

\begin{align}
 \langle \vec{y}, (\vec{A}^{-1}\vec{t})\rangle  =  \langle(\vec{A}^{-1})^{T}\vec{y},\vec{t} \rangle    \label{2.10}
\end{align}
Using \eqref{2.10} in \eqref{2.5} We can write,
\begin{align}
\implies \int_{\mathbb{R}^n} f(\vec{t}) e^{i \langle (\vec{A}^{-1})^{T}\vec{y},\vec{t} \rangle} \frac{d\vec{t}}{\left| \text{det}(\vec{A}) \right|}
\end{align}
replacing variable  $\vec{t}$ with $\vec{x}$.
\begin{align}
\implies \int_{\mathbb{R}^n} f(\vec{x}) e^{i \langle (\vec{A}^{-1})^{T}\vec{y},\vec{x} \rangle} \frac{d\vec{x}}{\left| \text{det}(\vec{A}) \right|} \label{2.11}
\end{align}
Hence option $1$ is correct.
\section{Example}
Let consider a matrix $\vec{A}$ and $\vec{x}$ as :
\begin{align}
\vec{A} = \myvec{x_1 & 2x_2 \\ 3x_1  & 4x_2}, \vec{x} = \myvec{x_1 \\x_2}\\
\vec{A}\vec{x} = \vec{t}\\
\implies \vec{x} = \vec{A}^{-1}\vec{t}\\
\vec{A^{-1}} =\myvec{-\frac{2}{x_1} & \frac{1}{x_1} \\ \frac{3}{2x_2} & -\frac{1}{2x_2}}\\
\vec{t} = \myvec{x_1^2 + 2x_2^2 \\ 3x_1^2 + 4x_2^2}\\
\intertext{Let consider another matrix $\vec{y}$ as : }
\vec{y} = \myvec{5x_1 \\7x_2}
\end{align}
Now,
\begin{multline}
\langle \vec{y},(\vec{A^{-1}}\vec{t}) \rangle =\vec{y}^T(\vec{A}^{-1}\vec{t}) = \myvec{5x_1 & 6{x_2}}\myvec{x_1 \\ x_2}  = 5x_1^2 + 6{x_2}^2 \label{3.7}
\end{multline}
And,
\begin{multline}
\langle   (\vec{A}^{-1})^T\vec{y}, \vec{t} \rangle = ((\vec{A}^{-1})^T\vec{y})^T \vec{t} = \myvec{-1 & 2}\myvec{x_1^2 +2{x_2}^2 \\3x_1^2 +4{x_2}^2}\\ = 5x_1^2 + 6{x_2}^2 \label{3.8}
\end{multline}
Hence from \eqref{3.7} and \eqref{3.8} we can conclude,
\begin{align}
 \langle \vec{y},(\vec{A^{-1}}\vec{t}) \rangle = \langle (\vec{A}^{-1})^T\vec{y}, \vec{t} \rangle
\end{align}
 We have also, 
 \begin{align}
 d\vec{x} = \frac{d\vec{t}}{  \left | -2x_1x_2 \right|}
 \end{align}
 Hence our equation \eqref{2.11} becomes,
 \begin{align}
 \int_{\mathbb{R}^2} f(\vec{x}) e^{i(5x_1^2 + 6{x_2}^2)}\frac{d\vec{x}}{  \left | -2x_1x_2 \right|}
 \end{align}
\begin{align}
\implies  \int f(\vec{x}) e^{i(5x_1^2 + 6{x_2}^2)}\frac{1}{  \left | -2x_1x_2 \right|}d\myvec{x_1 \\ x_2}
\end{align}
\begin{align}
\implies  \int f(\vec{x}) e^{i(5x_1^2 + 6{x_2}^2)}\frac{1}{  \left | -2x_1x_2 \right|}\myvec{dx_1 \\ dx_2}
\end{align}


\end{document}