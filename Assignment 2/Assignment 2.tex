\documentclass[journal,12pt,twocolumn]{IEEEtran}
%
\usepackage{setspace}
\usepackage{gensymb}

\singlespacing
\usepackage[cmex10]{amsmath}
\usepackage{amsthm}

\usepackage{mathrsfs}
\usepackage{txfonts}
\usepackage{stfloats}
\usepackage{bm}
\usepackage{cite}
\usepackage{cases}
\usepackage{subfig}

\usepackage{longtable}
\usepackage{multirow}
%\usepackage{algorithm}
\usepackage{enumitem}
\usepackage{mathtools}
\usepackage{steinmetz}
\usepackage{tikz}
\usepackage{circuitikz}
\usepackage{verbatim}
\usepackage{tfrupee}
\usepackage[breaklinks=true]{hyperref}
%\usepackage{stmaryrd}
\usepackage{tkz-euclide} % loads  TikZ and tkz-base
%\usetkzobj{all}
\usetikzlibrary{calc,math}
\usepackage{listings}
    \usepackage{color}                                            %%
    \usepackage{array}                                            %%
    \usepackage{longtable}                                        %%
    \usepackage{calc}                                             %%
    \usepackage{multirow}                                         %%
    \usepackage{hhline}                                           %%
    \usepackage{ifthen}                                           %%
    \usepackage{lscape}     
\usepackage{multicol}
\usepackage{chngcntr}
%\usepackage{enumerate}

%\usepackage{wasysym}
%\newcounter{MYtempeqncnt}
\DeclareMathOperator*{\Res}{Res}
%\renewcommand{\baselinestretch}{2}
\renewcommand\thesection{\arabic{section}}
\renewcommand\thesubsection{\thesection.\arabic{subsection}}
\renewcommand\thesubsubsection{\thesubsection.\arabic{subsubsection}}

\renewcommand\thesectiondis{\arabic{section}}
\renewcommand\thesubsectiondis{\thesectiondis.\arabic{subsection}}
\renewcommand\thesubsubsectiondis{\thesubsectiondis.\arabic{subsubsection}}

% correct bad hyphenation here
\hyphenation{op-tical net-works semi-conduc-tor}
\def\inputGnumericTable{}                                 %%

\lstset{
%language=C,
frame=single, 
breaklines=true,
columns=fullflexible
}
%\lstset{
%language=tex,
%frame=single, 
%breaklines=true
%}

\begin{document}
%


\newtheorem{theorem}{Theorem}[section]
\newtheorem{problem}{Problem}
\newtheorem{proposition}{Proposition}[section]
\newtheorem{lemma}{Lemma}[section]
\newtheorem{corollary}[theorem]{Corollary}
\newtheorem{example}{Example}[section]
\newtheorem{definition}[problem]{Definition}

\newcommand{\BEQA}{\begin{eqnarray}}
\newcommand{\EEQA}{\end{eqnarray}}
\newcommand{\define}{\stackrel{\triangle}{=}}
\bibliographystyle{IEEEtran}
%\bibliographystyle{ieeetr}
\providecommand{\mbf}{\mathbf}
\providecommand{\pr}[1]{\ensuremath{\Pr\left(#1\right)}}
\providecommand{\qfunc}[1]{\ensuremath{Q\left(#1\right)}}
\providecommand{\sbrak}[1]{\ensuremath{{}\left[#1\right]}}
\providecommand{\lsbrak}[1]{\ensuremath{{}\left[#1\right.}}
\providecommand{\rsbrak}[1]{\ensuremath{{}\left.#1\right]}}
\providecommand{\brak}[1]{\ensuremath{\left(#1\right)}}
\providecommand{\lbrak}[1]{\ensuremath{\left(#1\right.}}
\providecommand{\rbrak}[1]{\ensuremath{\left.#1\right)}}
\providecommand{\cbrak}[1]{\ensuremath{\left\{#1\right\}}}
\providecommand{\lcbrak}[1]{\ensuremath{\left\{#1\right.}}
\providecommand{\rcbrak}[1]{\ensuremath{\left.#1\right\}}}
\theoremstyle{remark}
\newtheorem{rem}{Remark}
\newcommand{\sgn}{\mathop{\mathrm{sgn}}}
\providecommand{\abs}[1]{\left\vert#1\right\vert}
\providecommand{\res}[1]{\Res\displaylimits_{#1}} 
\providecommand{\norm}[1]{\left\lVert#1\right\rVert}
%\providecommand{\norm}[1]{\lVert#1\rVert}
\providecommand{\mtx}[1]{\mathbf{#1}}
\providecommand{\mean}[1]{E\left[ #1 \right]}
\providecommand{\fourier}{\overset{\mathcal{F}}{ \rightleftharpoons}}
%\providecommand{\hilbert}{\overset{\mathcal{H}}{ \rightleftharpoons}}
\providecommand{\system}{\overset{\mathcal{H}}{ \longleftrightarrow}}
	%\newcommand{\solution}[2]{\textbf{Solution:}{#1}}
\newcommand{\solution}{\noindent \textbf{Solution: }}
\newcommand{\cosec}{\,\text{cosec}\,}
\providecommand{\dec}[2]{\ensuremath{\overset{#1}{\underset{#2}{\gtrless}}}}
\newcommand{\myvec}[1]{\ensuremath{\begin{pmatrix}#1\end{pmatrix}}}
\newcommand{\mydet}[1]{\ensuremath{\begin{vmatrix}#1\end{vmatrix}}}
%\numberwithin{equation}{section}
\numberwithin{equation}{subsection}
%\numberwithin{problem}{section}
%\numberwithin{definition}{section}
\makeatletter
\@addtoreset{figure}{problem}
\makeatother
\let\StandardTheFigure\thefigure
\let\vec\mathbf
%\renewcommand{\thefigure}{\theproblem.\arabic{figure}}
\renewcommand{\thefigure}{\theproblem}
\def\putbox#1#2#3{\makebox[0in][l]{\makebox[#1][l]{}\raisebox{\baselineskip}[0in][0in]{\raisebox{#2}[0in][0in]{#3}}}}
     \def\rightbox#1{\makebox[0in][r]{#1}}
     \def\centbox#1{\makebox[0in]{#1}}
     \def\topbox#1{\raisebox{-\baselineskip}[0in][0in]{#1}}
     \def\midbox#1{\raisebox{-0.5\baselineskip}[0in][0in]{#1}}
\vspace{3cm}
\title{ Matrix Theory : Assignment 2 }
\author{Ritesh Kumar \\ Roll No: EE20RESCH11005}


\maketitle
\newpage
%\tableofcontents
\bigskip
\renewcommand{\thefigure}{\theenumi}
\renewcommand{\thetable}{\theenumi}
\begin{abstract}
This is a problem to balance a chemical equation using system of linear equations.
\end{abstract}
Download all  codes from 
\begin{lstlisting}
https://github.com/Debolena/EE5609/blob/master/Assignment_2/assignment_2.py
\end{lstlisting}
%

\section{Problem}
    Balance the following chemical equation.
    \begin{align}
        \label{eq1} \text{Zinc + Silver nitrate} \to \text{Zinc nitrate + Silver}
    \end{align}
    
\section{Solution}

Equation \ref{eq1}  can be written as :

\begin{align}
\label{eq2} Zn+ AgNO_{3} \to Ag + Zn(NO_{3})_{2}
\end{align}

Suppose balance form of the equation is :
\begin{align}
    \label{eq3} x_{1}Zn+ x_{2}AgNo_{3} \to x_{3}Ag + x_{4} Zn(NO_{3})_{2}
\end{align}
which results in the following equations:
\begin{align}
    ( x_{1} - 2x_{4} ) Zn = 0\\
    ( x_{2} - x_{3} ) Ag = 0\\
    ( x_{3} - 2 x_{4} ) N =0\\
    ( 3x_{3} - 6x_{4} ) O = 0
\end{align}
which can be expressed as
\begin{align}
    x_{1} + 0.x_{2} + 0.x_{3} - x_{4} = 0\\
    0.x_{1} + x_{2} - x_{3} + 0.x_{4} = 0\\
    0.x_{1} + 0.x_{2} + x_{3} - 2x_{4} =0\\
    0.x_{1} + 0.x_{2} + 3x_{3} - 6x_{4}= 0
\end{align}
resulting in the matrix equation
\begin{align}
    \label{eq: matrix}
    \myvec{1 & 0 & 0 & -1\\
           0 & 1 & -1 & 0\\
           0 & 0 & 1 & -2\\
           0 & 0 & 3 & -6 }\vec{X}
           =\vec{0}
\end{align}
where,
\begin{align}
   \vec{X}= \myvec{x_{1}\\x_{2}\\x_{3}\\x_{4}}
\end{align}
(\ref{eq: matrix}) can be reduced as follows:
\begin{align}
    \myvec{1 & 0 & 0 & -1\\
   	0 & 1 & -1 & 0\\
   	0 & 0 & 1 & -2\\
   	0 & 0 & 3 & -6 }
    \xleftrightarrow {R_{4}\leftarrow R_4 - 3R_3}
      \myvec{1 & 0 & 0 & -1\\
    	0 & 1 & -1 & 0\\
    	0 & 0 & 1 & -2\\
    	0 & 0 & 0 & 0 }
    \end{align}
Thus,
\begin{align}
    x_1 = x_4, x_ 2 = 2x_4, x_3 = 2x_4\\
    \implies \quad\vec{X } = \myvec{x_{4} \\ 2x_{4}\\ 2x_{4} \\ x_{4}}   =  x_4 \myvec{1\\ 2 \\ 2 \\1}
    \end{align} 
by substituting $x_4= 1$, we get :
\begin{align}
 \implies \quad\vec{X} = \myvec{1\\ 2 \\ 2 \\1}
\end{align}

\hfill\break
%\vspace{5mm} 
Hence, (\ref{eq3}) finally becomes
\begin{align}
Zn + 2AgNo_{3} \to 2Ag + Zn(NO_{3})_{2}
\end{align}
\end{document}