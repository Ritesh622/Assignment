\documentclass[10pt, twocolumn]{article}
\usepackage[margin=1in,left=1.5in,includefoot]{geometry}
\usepackage{cite}
 \usepackage{amsmath,amssymb,amsfonts}
\usepackage{algorithmic}
\usepackage{verbatimbox}
\usepackage{graphicx}
\usepackage{textcomp}
\usepackage{xcolor}
\usepackage{amsmath}
\usepackage[english]{algorithm2e}
\usepackage{setspace}
%\usepackage{subfig} 
\usepackage{array}
 \usepackage{stfloats}
% \usepackage{float} 
\usepackage{subfloat}
\usepackage{subcaption}
\usepackage{array}
%\newcolumntype{P}[1]{>{\centering\arraybackslash}p{#1}}
\begin{document}
	
	\begin{titlepage}
		\begin{center}
			\vspace*{1cm}
			
			\textbf{ \huge{Assignment 1}}
			\vspace{1.5cm}
			
			\textbf{Ritesh Kumar} \\
			\textbf{(EE20RESCH1105)}\\
			\textbf{Communication and Signal Processing}
			\date{Today}
			
		\end{center}
	\end{titlepage}
\section{Abstract}
This document demonstrate to find  the ratio of line segment if a line joining by two points and is divided by another line.	
\hspace{10cm}

\begin{center}
{\textbf{Problem Statement}} \\	
\end{center}
\hspace{4cm}
In what ratio is the line joining   $ \bigl(\begin{smallmatrix}
-1\\ 
1 \\
\end{smallmatrix}\bigr)$ and $ \bigl(\begin{smallmatrix}
	5\\ 
	7 \\
\end{smallmatrix}\bigr) $ divided by the line $ \begin{bmatrix}
1 &  1
\end{bmatrix}x = 4$
	
	
	\section{Theory}
	We can solve these types of problems using following method : \\
Using  basic coordinate geometary i.e division of a line by another line. We can find  point of intersection of two line and the by using basic division formula we can find the ratio.
	Suppose a point \textbf{P (x,y)} is lies on  line  and divide the line in the ratio of m:n then, \\
	\begin{equation}
	 x = \frac{x_{2}m + x_{1}n}{m+n}
\end{equation}
\begin{equation}
 y = \frac{y_{2}m + y_{1}n}{m+n}
\end{equation}
	  
	 We can find the equation of line using two point formula given as: \\
	 \begin{equation}
\frac{y - y_{1}}{x - x_{1} } = \frac{y_{2} - y_{1}}{x_{2} - x_{1} } 
	 \end{equation}

With above equations we	 can have two equation and two unknown and we can solve the quadratic equation and get the appropriate ration. 

\section{Solution}
 Using equation	3 :\\
	 
$ \frac{y - 1}{x + 1 } = \frac{7 - 1}{ 5 +1 } $ \\ \\
$ \Rightarrow  (y - 1 ) 6 = 6 ( x +1 ) 	$  \\ \\
$\Rightarrow  y - x = 2 	$  \\
Hence,
\begin{equation}
 y - x = 2 	 
\end{equation}
And, \\

 \begin{equation}
   y + x = 4 	 
 \end{equation}
 Solving above these two equations, we get $ x = 1 $ and, $ y = 3 $

Now, applying section formula on line $ y - x = 2$,

Let $\frac{m}{n}$ = k , then,
\begin{subequations}
\begin{equation}
\frac{kx_{2} +x_{1}}{k + 1} = x
\end{equation}
\begin{equation}
\frac{5k -1 }{k + 1} = 1
\end{equation}
$\Rightarrow$ k = $ - \frac{1}{2}$ ,
Similarly,

\begin{equation}
\frac{ky_{2} +y_{1}}{k + 1} = y
\end{equation}
\begin{equation}
\frac{7k  + 1 }{k + 1} = 3 
\end{equation}
 $\Rightarrow$ k = $ - \frac{1}{2}$
\end{subequations}

\begin{center}
	\begin{figure}[htb!]
		\centering
		\includegraphics[width=.50\textwidth, height=.30\textheight]{Assignment_1.jpg}
		\caption{Intersection of two lines}
		\label{fig1}
	\end{figure}
\end{center}
Hence, the  line joining  $ \bigl(\begin{smallmatrix}
-1\\ 
1 \\
\end{smallmatrix}\bigr)$ and $ \bigl(\begin{smallmatrix}
5\\ 
7 \\
\end{smallmatrix}\bigr) $ will be  divided by the line $ \begin{bmatrix}
1 &  1
\end{bmatrix}x = 4$ in the ration of 1 : 2.
\end{document}