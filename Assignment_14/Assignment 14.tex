\documentclass[journal,12pt,twocolumn]{IEEEtran}
\usepackage{setspace}
\usepackage{gensymb}
\singlespacing
\usepackage[cmex10]{amsmath}
\usepackage{amsthm}
\usepackage{mathrsfs}
\usepackage{txfonts}
\usepackage{stfloats}
\usepackage{bm}
\usepackage[table,xcdraw]{xcolor}
\usepackage{cite}
\usepackage{cases}
\usepackage{subfig}
\usepackage{longtable}
\usepackage{multirow}
%\usepackage{algorithm}
\usepackage{enumitem}
\usepackage{mathtools}
\usepackage{steinmetz}
\usepackage{tikz}
\usepackage{circuitikz}
\usepackage{verbatim}
\usepackage{tfrupee}
\usepackage[breaklinks=true]{hyperref}
%\usepackage{stmaryrd}
\usepackage{tkz-euclide} % loads  TikZ and tkz-base
%\usetkzobj{all}
\usetikzlibrary{calc,math}
\usepackage{listings}
\usepackage{color}                                            %%
\usepackage{array}                                            %%
\usepackage{longtable}                                        %%
\usepackage{calc}                                             %%
\usepackage{multirow}                                         %%
\usepackage{hhline}                                           %%
\usepackage{ifthen}                                           %%
%optionally (for landscape tables embedded in another document): %%
\usepackage[]{caption}
\usepackage{lscape}     
\usepackage{multicol}
\usepackage{chngcntr}
\usepackage{enumerate}
%\usepackage{placeins}
%\usepackage{float}
%\restylefloat{table}
%\usepackage{wasysym}
%\newcounter{MYtempeqncnt}
\DeclareMathOperator*{\Res}{Res}
%\renewcommand{\baselinestretch}{2}
\renewcommand\thesection{\arabic{section}}
\renewcommand\thesubsection{\thesection.\arabic{subsection}}
\renewcommand\thesubsubsection{\thesubsection.\arabic{subsubsection}}
\newcommand\numberthis{\addtocounter{equation}{1}\tag{\theequation}}
\renewcommand\thesectiondis{\arabic{section}}
\renewcommand\thesubsectiondis{\thesectiondis.\arabic{subsection}}
\renewcommand\thesubsubsectiondis{\thesubsectiondis.\arabic{subsubsection}}
\newcommand\bigSigma[1][17]{\mbox{\fontsize{#1}{0}\selectfont$\Sigma$}}
% correct bad hyphenation here
\hyphenation{op-tical net-works semi-conduc-tor}
\def\inputGnumericTable{}                                 %%

\lstset{
	%language=C,
	frame=single, 
	breaklines=true,
	columns=fullflexible
}


\begin{document}
	%
	
	
	\newtheorem{theorem}{Theorem}[section]
	\newtheorem{problem}{Problem}
	\newtheorem{proposition}{Proposition}[section]
	\newtheorem{lemma}{Lemma}[section]
	\newtheorem{corollary}[theorem]{Corollary}
	\newtheorem{example}{Example}[section]
	\newtheorem{definition}[problem]{Definition}
	\newcommand\bigZero[1][17]{\mbox{\fontsize{#1}{0}\selectfont$0$}}
	\newcommand{\BEQA}{\begin{eqnarray}}
	\newcommand{\EEQA}{\end{eqnarray}}
	\newcommand{\define}{\stackrel{\triangle}{=}}
	\bibliographystyle{IEEEtran}
	%\bibliographystyle{ieeetr}
	\providecommand{\mbf}{\mathbf}
	\providecommand{\pr}[1]{\ensuremath{\Pr\left(#1\right)}}
	\providecommand{\qfunc}[1]{\ensuremath{Q\left(#1\right)}}
	\providecommand{\sbrak}[1]{\ensuremath{{}\left[#1\right]}}
	\providecommand{\lsbrak}[1]{\ensuremath{{}\left[#1\right.}}
	\providecommand{\rsbrak}[1]{\ensuremath{{}\left.#1\right]}}
	\providecommand{\brak}[1]{\ensuremath{\left(#1\right)}}
	\providecommand{\lbrak}[1]{\ensuremath{\left(#1\right.}}
	\providecommand{\rbrak}[1]{\ensuremath{\left.#1\right)}}
	\providecommand{\cbrak}[1]{\ensuremath{\left\{#1\right\}}}
	\providecommand{\lcbrak}[1]{\ensuremath{\left\{#1\right.}}
	\providecommand{\rcbrak}[1]{\ensuremath{\left.#1\right\}}}
	\theoremstyle{remark}
	\newtheorem{rem}{Remark}
	\newcommand{\sgn}{\mathop{\mathrm{sgn}}}
	\providecommand{\abs}[1]{\left\vert#1\right\vert}
	\providecommand{\res}[1]{\Res\displaylimits_{#1}} 
	\providecommand{\norm}[1]{\left\lVert#1\right\rVert}
	%\providecommand{\norm}[1]{\lVert#1\rVert}
	\providecommand{\mtx}[1]{\mathbf{#1}}
	\providecommand{\mean}[1]{E\left[ #1 \right]}
	\providecommand{\fourier}{\overset{\mathcal{F}}{ \rightleftharpoons}}
	%\providecommand{\hilbert}{\overset{\mathcal{H}}{ \rightleftharpoons}}
	\providecommand{\system}{\overset{\mathcal{H}}{ \longleftrightarrow}}
	%\newcommand{\solution}[2]{\textbf{Solution:}{#1}}
	\newcommand{\solution}{\noindent \textbf{Solution: }}
	\newcommand{\cosec}{\,\text{cosec}\,}
	\providecommand{\dec}[2]{\ensuremath{\overset{#1}{\underset{#2}{\gtrless}}}}
	\newcommand{\myvec}[1]{\ensuremath{\begin{pmatrix}#1\end{pmatrix}}}
	\newcommand{\mydet}[1]{\ensuremath{\begin{vmatrix}#1\end{vmatrix}}}
	\numberwithin{equation}{subsection}
	\makeatletter
	\@addtoreset{figure}{problem}
	\makeatother
	\let\StandardTheFigure\thefigure
	\let\vec\mathbf
	\renewcommand{\thefigure}{\theproblem}
	\def\putbox#1#2#3{\makebox[0in][l]{\makebox[#1][l]{}\raisebox{\baselineskip}[0in][0in]{\raisebox{#2}[0in][0in]{#3}}}}
	\def\rightbox#1{\makebox[0in][r]{#1}}
	\def\centbox#1{\makebox[0in]{#1}}
	\def\topbox#1{\raisebox{-\baselineskip}[0in][0in]{#1}}
	\def\midbox#1{\raisebox{-0.5\baselineskip}[0in][0in]{#1}}
	\vspace{3cm}
	\title{Matrix Theory Assignment 14}
	\author{Ritesh Kumar \\ EE20RESCH11005}
	
	
	\maketitle
	\newpage
	%\tableofcontents
	\bigskip
	\renewcommand{\thefigure}{\theenumi}
	\renewcommand{\thetable}{\theenumi}
	\counterwithout{figure}{section}
	\counterwithout{figure}{subsection}
	\date{Today}
	
	
	\begin{abstract}
		This problem is all about to  introducing the concept of characteristic polynomial over a filed.
	\end{abstract}
	All the codes for this document can be found at
	\begin{lstlisting}
	https://github.com/Ritesh622/Assignment_EE5609/tree/master/Assignment_14
	\end{lstlisting}
	\section{Problem}
	Let $\mathbf{V}$ be a real vector space and E an idempotent linear operator on $\mathbf{V}$, i.e., a projection. Prove that $\myvec{\vec{I} + \vec{E}}$ is
	invertible. Find $\myvec{\vec{I} + \vec{E}}^{-1}.$
	
	\section{solution}
	we have $\vec{E}$ and it  is idempotent. And we know that the eigen value of idempotent matrix is either 0 or 1.
	When we add the identity matrix in this:
	\begin{align}
	\vec{I} + \vec{E}
	\end{align}
	Then eigen value will be either 1 or 2. Hence $\myvec{\vec{I} + \vec{E}}$ is invertible.
	Since $\vec{E}$ is an idempotent matrix, that is :
	\begin{align}
	\vec{E}^2 = \vec{E} \label{2.1}
	\intertext{Let,}
	\vec{A} = \vec{I} + \vec{E} \label{2.3}\\  
	\implies \vec{E} = \vec{A} - \vec{I}\label{2.4}
	\end{align}
	\begin{align}
	\implies \vec{E}^2 = (\vec{A} - \vec{I})(\vec{A} - \vec{I})= \vec{A}^2 - 2\vec{A} +\vec{I}^2
	\end{align}
	\begin{align}
	\intertext{From \ref{2.1},}
	\implies  \vec{E} = \vec{A}^2 -2\vec{A} + \vec{I}
	\end{align}
	\begin{align}
	\intertext{Using \eqref{2.4} we have,}
	\implies \vec{A} - \vec{I} = \vec{A}^2 -2\vec{A} +\vec{I} = \vec{A}^2 -3\vec{A} + 2\vec{I} = 0
	\end{align}
	\begin{align}
	\implies \vec{I} = \frac{ 3\vec{A} - \vec{A}^2}{2}
	\end{align}
	multiplying $\vec{A}^{-1}$ both side,
	\begin{align}
	\vec{A}^{-1} = \frac{3\vec{I}-\vec{A}}{2} = \frac{3\vec{I} - (\vec{I} +\vec{E})}{2}\\
\implies \vec{A^{-1}} = \frac{2\vec{I}- \vec{E}}{2}\\
	\intertext{Using \eqref{2.3}, we  have,}
(\vec{I} + \vec{E})^{-1} =  \vec{I}- \frac{1}{2}\vec{E} \label{2.11}
	\end{align}

	
	\section{example}
	Let consider a matrix $\vec{E}$ as :
	\begin{align}
	\vec{E} =  \myvec{1 & 0 \\ 0 & 0}\\
	\implies \vec{E}^2 = \myvec{1 & 0 \\ 0 & 0}\myvec{1 & 0 \\ 0 & 0} = \myvec{1 & 0 \\ 0 & 0}\\
\implies \vec{E}^2 = E.\\
\intertext{Now,}
\vec{I}+ \vec{E} = \myvec{1 & 0 \\ 0 & 1} + \myvec{ 1 & 0 \\ 0 & 0}\\
\implies \vec{I} + \vec{E} = \myvec{2 & 0 \\ 0 & 1} = \myvec{2 & 0 \\ 0 &1} \label{3.8}\\
\end{align}	
Now let find the eigen value of matrix $\myvec{\vec{I} + \vec{E}}$ :
\begin{align}
\implies \myvec{2 - \l  & 0 \\ 0  &  1 - \lambda} = 0\\
\implies (2-\lambda)(1 - \lambda) = 0\\
\implies \lambda_1 = 2, \lambda_2 = 1 \label{3.16}
\end{align}
 
The eigen values of the matrix $(\vec{I} + \vec{E})$ from \eqref{3.16} are  2 and 1. Since none of the eigen value is zero, hence matrix is invertible.\\
Inverse of the matrix  from \eqref{2.11} is :
\begin{align}
  \myvec{\vec{I} + \vec{E}}^{-1} =\myvec{1 & 0 \\ 0 & 1} +  \myvec{ -\frac{1}{2} & 0 \\ 0 & 1} \label{3.19}
\end{align}
	
\end{document}