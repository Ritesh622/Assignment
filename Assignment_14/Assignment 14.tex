\documentclass[journal,12pt,twocolumn]{IEEEtran}
\usepackage{setspace}
\usepackage{gensymb}
\singlespacing
\usepackage[cmex10]{amsmath}
\usepackage{amsthm}
\usepackage{mathrsfs}
\usepackage{txfonts}
\usepackage{stfloats}
\usepackage{bm}
\usepackage[table,xcdraw]{xcolor}
\usepackage{cite}
\usepackage{cases}
\usepackage{subfig}
\usepackage{longtable}
\usepackage{multirow}
%\usepackage{algorithm}
\usepackage{enumitem}
\usepackage{mathtools}
\usepackage{steinmetz}
\usepackage{tikz}
\usepackage{circuitikz}
\usepackage{verbatim}
\usepackage{tfrupee}
\usepackage[breaklinks=true]{hyperref}
%\usepackage{stmaryrd}
\usepackage{tkz-euclide} % loads  TikZ and tkz-base
%\usetkzobj{all}
\usetikzlibrary{calc,math}
\usepackage{listings}
\usepackage{color}                                            %%
\usepackage{array}                                            %%
\usepackage{longtable}                                        %%
\usepackage{calc}                                             %%
\usepackage{multirow}                                         %%
\usepackage{hhline}                                           %%
\usepackage{ifthen}                                           %%
%optionally (for landscape tables embedded in another document): %%
\usepackage[]{caption}
\usepackage{lscape}     
\usepackage{multicol}
\usepackage{chngcntr}
\usepackage{enumerate}
%\usepackage{placeins}
%\usepackage{float}
%\restylefloat{table}
%\usepackage{wasysym}
%\newcounter{MYtempeqncnt}
\DeclareMathOperator*{\Res}{Res}
%\renewcommand{\baselinestretch}{2}
\renewcommand\thesection{\arabic{section}}
\renewcommand\thesubsection{\thesection.\arabic{subsection}}
\renewcommand\thesubsubsection{\thesubsection.\arabic{subsubsection}}
\newcommand\numberthis{\addtocounter{equation}{1}\tag{\theequation}}
\renewcommand\thesectiondis{\arabic{section}}
\renewcommand\thesubsectiondis{\thesectiondis.\arabic{subsection}}
\renewcommand\thesubsubsectiondis{\thesubsectiondis.\arabic{subsubsection}}
\newcommand\bigSigma[1][17]{\mbox{\fontsize{#1}{0}\selectfont$\Sigma$}}
% correct bad hyphenation here
\hyphenation{op-tical net-works semi-conduc-tor}
\def\inputGnumericTable{}                                 %%

\lstset{
	%language=C,
	frame=single, 
	breaklines=true,
	columns=fullflexible
}


\begin{document}
	%
	
	
	\newtheorem{theorem}{Theorem}[section]
	\newtheorem{problem}{Problem}
	\newtheorem{proposition}{Proposition}[section]
	\newtheorem{lemma}{Lemma}[section]
	\newtheorem{corollary}[theorem]{Corollary}
	\newtheorem{example}{Example}[section]
	\newtheorem{definition}[problem]{Definition}
	\newcommand\bigZero[1][17]{\mbox{\fontsize{#1}{0}\selectfont$0$}}
	\newcommand{\BEQA}{\begin{eqnarray}}
	\newcommand{\EEQA}{\end{eqnarray}}
	\newcommand{\define}{\stackrel{\triangle}{=}}
	\bibliographystyle{IEEEtran}
	%\bibliographystyle{ieeetr}
	\providecommand{\mbf}{\mathbf}
	\providecommand{\pr}[1]{\ensuremath{\Pr\left(#1\right)}}
	\providecommand{\qfunc}[1]{\ensuremath{Q\left(#1\right)}}
	\providecommand{\sbrak}[1]{\ensuremath{{}\left[#1\right]}}
	\providecommand{\lsbrak}[1]{\ensuremath{{}\left[#1\right.}}
	\providecommand{\rsbrak}[1]{\ensuremath{{}\left.#1\right]}}
	\providecommand{\brak}[1]{\ensuremath{\left(#1\right)}}
	\providecommand{\lbrak}[1]{\ensuremath{\left(#1\right.}}
	\providecommand{\rbrak}[1]{\ensuremath{\left.#1\right)}}
	\providecommand{\cbrak}[1]{\ensuremath{\left\{#1\right\}}}
	\providecommand{\lcbrak}[1]{\ensuremath{\left\{#1\right.}}
	\providecommand{\rcbrak}[1]{\ensuremath{\left.#1\right\}}}
	\theoremstyle{remark}
	\newtheorem{rem}{Remark}
	\newcommand{\sgn}{\mathop{\mathrm{sgn}}}
	\providecommand{\abs}[1]{\left\vert#1\right\vert}
	\providecommand{\res}[1]{\Res\displaylimits_{#1}} 
	\providecommand{\norm}[1]{\left\lVert#1\right\rVert}
	%\providecommand{\norm}[1]{\lVert#1\rVert}
	\providecommand{\mtx}[1]{\mathbf{#1}}
	\providecommand{\mean}[1]{E\left[ #1 \right]}
	\providecommand{\fourier}{\overset{\mathcal{F}}{ \rightleftharpoons}}
	%\providecommand{\hilbert}{\overset{\mathcal{H}}{ \rightleftharpoons}}
	\providecommand{\system}{\overset{\mathcal{H}}{ \longleftrightarrow}}
	%\newcommand{\solution}[2]{\textbf{Solution:}{#1}}
	\newcommand{\solution}{\noindent \textbf{Solution: }}
	\newcommand{\cosec}{\,\text{cosec}\,}
	\providecommand{\dec}[2]{\ensuremath{\overset{#1}{\underset{#2}{\gtrless}}}}
	\newcommand{\myvec}[1]{\ensuremath{\begin{pmatrix}#1\end{pmatrix}}}
	\newcommand{\mydet}[1]{\ensuremath{\begin{vmatrix}#1\end{vmatrix}}}
	\numberwithin{equation}{subsection}
	\makeatletter
	\@addtoreset{figure}{problem}
	\makeatother
	\let\StandardTheFigure\thefigure
	\let\vec\mathbf
	\renewcommand{\thefigure}{\theproblem}
	\def\putbox#1#2#3{\makebox[0in][l]{\makebox[#1][l]{}\raisebox{\baselineskip}[0in][0in]{\raisebox{#2}[0in][0in]{#3}}}}
	\def\rightbox#1{\makebox[0in][r]{#1}}
	\def\centbox#1{\makebox[0in]{#1}}
	\def\topbox#1{\raisebox{-\baselineskip}[0in][0in]{#1}}
	\def\midbox#1{\raisebox{-0.5\baselineskip}[0in][0in]{#1}}
	\vspace{3cm}
	\title{Matrix Theory Assignment 14}
	\author{Ritesh Kumar \\ EE20RESCH11005}
	
	
	\maketitle
	\newpage
	%\tableofcontents
	\bigskip
	\renewcommand{\thefigure}{\theenumi}
	\renewcommand{\thetable}{\theenumi}
	\counterwithout{figure}{section}
	\counterwithout{figure}{subsection}
	\date{Today}
	
	
	\begin{abstract}
		This problem is all about to  introducing the concept of characteristic polynomial over a filed.
	\end{abstract}
	All the codes for this document can be found at
	\begin{lstlisting}
	https://github.com/Ritesh622/Assignment_EE5609/tree/master/Assignment_14
	\end{lstlisting}
	\section{Problem}
	Let $\mathbf{V}$ be a real vector space and E an idempotent linear operator on $\mathbf{V}$, i.e., a projection. Prove that $\myvec{\vec{I} + \vec{E}}$ is
	invertible. Find $\myvec{\vec{I} + \vec{E}}^{-1}.$
	
	\section{solution}
	Since $\vec{E}$ is an idempotent matrix, that is :
	\begin{align}
	\vec{E}^2 = \vec{E} \label{2.1}
	\end{align}
	Hence it will satisfy the polynomial,
	\begin{align}
	x^2-x = 0
	\end{align}
	Thus minimal polynomial will be,
	\begin{align}
	m_{\vec{E}}(x) = x^2-x = 0\\
	\implies m_{\vec{E}}(x) = x(x-1) = 0\\
	\intertext{Hence minimal polynomial $m_{\vec{E}}(x)$, of $\vec{E}$ divides $x^2-x$ that is,}
	m_{\vec{E}}(x) = x \intertext{or,} m_{\vec{E}}(x) = x -1\\
	\intertext{if}
	m_{\vec{E}}(x) = x\\
	\implies \vec{E} = \vec{0}\\
	\intertext{if}
	m_{\vec{E}}(x) = x - 1\\
	\implies \vec{E} = \vec{I}
	\end{align}
	Hence, if $\vec{E}$ is idempotent  then, minimal polynomial of $\vec{E}$ is product of distinct polynomial of degree one.Thus matrix $\vec{E}$ is similar to diagonal matrix with diagonal entries consisting of characteristic value $0$ and $1$.\\
	Since $\vec{E}$ is diagonalizable, then there exist at least one basis such that :
	\begin{align}
	\boldsymbol{\ss} = \left \{  \beta_{1}, \beta_{2} \dots , \beta{n} \right \}
	\intertext{Such that}
	\vec{E}\beta_i = \beta_i, \forall i = 1,2 \dots k\\
	\intertext{and,}
	\vec{E}\beta_i = 0, \forall i = k+1, \dots n.\\
	\implies \myvec{ \vec{I} + \vec{E}}\beta_i =  2\beta_i, \forall i = 1,2 \dots k.
	\intertext{And,}
	\implies \myvec{ \vec{I} + \vec{E}}\beta_i =  \beta_i, \forall i = k+1 \dots n.
	\end{align} 
	In matrix form we can write it as:
	\begin{align}
	\left[ \vec{I +E}\right]_{\ss} =  \myvec{2\vec{I_1} & \vec{0}\\ \vec{0} & \vec{I_2}} \label{2.16}
	\end{align}
	Where, $\vec{I_{1}}$ is $k \times k$ and $\vec{I_{2}}$ is $({n-k}) \times ({n - k})$ identity matrices, and each $\vec{0}$ represents the zero matrix of
	appropriate dimension.\\
	From \refeq{2.16} we can calculate the determinant as:
	\begin{align}
	\text{det}\myvec{\vec{I} + \vec{E}} = 2^k \neq  0
	\end{align}
	From \ref{2.16} we can observe that, eigen values of  $ \left[\vec{I}+ \vec{E} \right] $ are k$^{th}$ number of 2 and $(n-k)^{th}$ number of 1. hence  none of the eigen value  of the matrix is zero. hence it is invertible. \\
	Since $\left[ \vec{I} + \vec{E}\right]_{\ss}$ is combination of identity matrices. Hence, from \eqref{2.16}, inverse of the matrix  $\left[ \vec{I} + \vec{E}\right]_{\ss}$ is given as :
	\begin{multline}
	\myvec{\left[ \vec{I} + \vec{E}\right]_{\ss} }^{-1} = \myvec{\frac{1}{2}\vec{I_1} & \vec{0} \\ \vec{0} & \vec{I_2} }\\
	=   \myvec{\vec{I_1} & \vec{0} \\
		\vec{0} & \vec{I_2} } +  \myvec{-\frac{1}{2}\vec{I_1} & \vec{0} \\
		\vec{0} & \vec{I_2}} = \vec{I} - \frac{1}{2}\left[\vec{E}\right]_{\ss}
	\end{multline}
	Hence,
	\begin{align}
	\myvec{\left[ \vec{I} + \vec{E}\right]_{\ss} }^{-1} = \vec{I} - \frac{1}{2}\left[\vec{E}\right]_{\ss}
	\end{align}
	we can also verify our results as:
	\begin{multline}
	\myvec{\vec{I} + \vec{E}}\myvec{\vec{I} - \frac{1}{2}\vec{E}} = \vec{I}^2- \frac{1}{2}\vec{E} +\vec{E}- \frac{1}{2}\vec{E}^2 \label{2.20}
	\end{multline}
	From \eqref{2.1}, We have $\vec{E}^2 = \vec{E}$, Hence \eqref{2.20} becomes,
	\begin{align}
	\vec{I} - \frac{1}{2}\vec{E} +\vec{E} + \frac{1}{2}\vec{E} = \vec{I}
	\end{align}
	
	\section{example}
	Let consider a matrix $\vec{E}$ as :
	\begin{align}
	\vec{E} =  \myvec{1 & 0 \\ 0 & 0}\\
	\implies \vec{E}^2 = \myvec{1 & 0 \\ 0 & 0}\myvec{1 & 0 \\ 0 & 0} = \myvec{1 & 0 \\ 0 & 0}\\
\implies \vec{E}^2 = E.\\
\intertext{Basis for this matix is}
\beta_1 = \myvec{1 \\ 0 } \text{and}, \beta_2 = \myvec{0 \\0}\\
\intertext{We have,} 
\vec{E}\beta_1 =\myvec{1 & 0 \\ 0 & 0} \myvec{1 \\ 0} = \myvec{1 \\ 0 } = \beta_1\\
\intertext{And,}
\vec{E}\beta_2 = \myvec{1 & 0 \\ 0 & 0}\myvec{0 \\ 0 }  =  \myvec{0 \\ 0}= \beta_2
\end{align}
\begin{align}
\intertext{Now,}
\vec{I}+ \vec{E} = \myvec{1 & 0 \\ 0 & 1} + \myvec{ 1 & 0 \\ 0 & 0}\\
\implies \vec{I} + \vec{E} = \myvec{2 & 0 \\ 0 & 1} = \myvec{2 & 0 \\ 0 &1} \label{3.8}\\
\implies \myvec{\vec{I} + \vec{E}} \beta_1 = \myvec{2 & 0 \\ 0 &1}\myvec{1 \\ 0} =  \myvec{2 \\ 0}\\
\implies \myvec{\vec{I} + \vec{E}} \beta_1 = 2 \myvec{1 \\ 0} = 2\beta_1\\
\intertext{Similarly,}
\implies \myvec{\vec{I} + \vec{E}} \beta_2 = \myvec{2 & 0 \\ 0 &1}\myvec{0 \\ 0} =  \myvec{0 \\ 0}\\
\implies \myvec{\vec{I} + \vec{E}} \beta_1 = 2 \myvec{0 \\ 0} = \beta_2\\
\implies \left[ \vec{I} + \vec{E}\right]_{\ss} = \myvec{2 & 0 \\ 0 & 1} = \myvec{2 \vec{I_1} & \vec{0} \\ \vec{0} & \vec{I_1}} \label{2.13}
\end{align}	
Now let find the eigen value of matrix $\myvec{\vec{I} + \vec{E}}$ :
\begin{align}
\implies \myvec{2 - \l  & 0 \\ 0  &  1 - \lambda} = 0\\
\implies (2-\lambda)(1 - \lambda) = 0\\
\implies \lambda_1 = 2, \lambda_2 = 1 \label{3.16}
\end{align}
 
The eigen values of the matrix $(\vec{I} + \vec{E})$ from \eqref{3.16} are  2 and 1. Since none of the eigen value is zero, hence matrix is invertible.

Here we have $k = 1$ and $ n = 2 $ and, $(n - k) = 1$. Hence size of  $\vec{I_1}$ and $\vec{I_2}$  are $ 1 \times 1$. Similarly, size of zero matrix is also $1 \times 1$.
Now determinant of $\vec{I} + \vec{E} $ is :



\begin{align}
\text{det}\myvec{\vec{I} + \vec{E}} = 2^k=  2 ^1 = 2 \neq 0.
\end{align}
Inverse of the matrix is :
\begin{align}
  \myvec{ \left[ \vec{I} + \vec{E}\right]}_{\ss}^{-1} = \myvec{\frac{1}{2} & 0 \\ 0 & 1}\\
\implies \myvec{\frac{1}{2} & 0 \\ 0 & 1}  =\myvec{1 & 0 \\ 0 & 1} +  \myvec{ -\frac{1}{2} & 0 \\ 0 & 1} \label{3.19}
\end{align}
\begin{align}
\intertext{Hence, equation \eqref{3.19} can be written as:}
   \myvec{ \left[ \vec{I} + \vec{E}\right]}_{\ss}^{-1}  = \myvec{\vec{I_1} & \vec{0} \\
	\vec{0} & \vec{I_2} } +  \myvec{-\frac{1}{2}\vec{I_1} & \vec{0} \\
	\vec{0} & \vec{I_2}} = \vec{I} - \frac{1}{2}\left[\vec{E}\right]_{\ss}
\end{align}



















	
	
\end{document}